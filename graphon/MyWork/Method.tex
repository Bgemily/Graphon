%!TEX root = Main.tex


\section{Method} \label{sec:method}

\subsection{Evaluating the distance function}
In practice, the distance  between a point process $N$ and a probability distribution function $f$ is 
evaluated by computing distance between their (smoothed) probability distribution functions, because the distance between cumulative distribution functions can be misleading in some certain cases. 
We illustrate this through figure \ref{pic: pdf better}. Although the black dash pdf is closer to the green pdf than the red pdf, it is closer to the red curve in terms of cdf.


\begin{figure}[H]
\includegraphics[width=0.49\textwidth]{../simulation/plots/pdf_vs_cdf_c}
\includegraphics[width=0.49\textwidth]{../simulation/plots/pdf_vs_cdf_d}
\caption{ Distance between cdfs is misleading. Top left: pdf to be shifted (black dash line) and target pdf (green and red solid line). Top right: corresponding cdf. Bottom left: distance between pdfs as a function of shift. Bottom right: distance between cdfs as a function of shift.}
\label{pic: pdf better}
\end{figure}


However, as shown in figure \ref{pic: cdf better}, aligning cdf can help us to locate the minimizer of distance function, especially when their is a flat area between two pdfs.

\begin{figure}[H]
\includegraphics[width=0.49\textwidth]{../simulation/plots/pdf_vs_cdf_a}
\includegraphics[width=0.49\textwidth]{../simulation/plots/pdf_vs_cdf_b}
\caption{ cdf can help to locate the minimizer. Top left: pdf to be shifted (dash line) and target pdf (solid line). Top right: corresponding cdfs. Bottom left: distance between pdfs as a function of shift. Bottom right: distance between cdfs as a function of shift.}
\label{pic: cdf better}
\end{figure}

Based on these properties of cdf and pdf, we align pdfs using gradient descent algorithm with initialization obtained by aligning their corresponding cdfs. 
We use the shape-invariant method proposed by \citet{Bigot2013}.
The idea is to find the optimal shift that aligns $f$ and $g$ best in the Fourier domain.
Let $\theta_j$ and $\gamma_j, j=-(N-1)/2,\cdots,(N-1)/2$ be the discrete Fourier coefficients of $f$ and $g$, where $N$ is the length of discretized $f$ and $g$. The time shift parameter $\tau$ can be estimated by solving the following problem by gradient descent

	\begin{equation}
	\begin{aligned}\label{eq:time shift}
	\hat n&=
	\underset{|n|\leq (N-1)/2}{\arg\min}
	\sum_{|j|\leq (N-1)/2}
	\left| \theta_{j}e^{\complexunit 2\pi j n/N}
	-\gamma_{j} \right|^2
	,
	\end{aligned}
	\end{equation}
where $n=N\cdot\tau/2T$.
Initialization is given by the alignment result of their cdfs.
For details see \ref{app}.

\noindent
[add details for obtaining smooth pdf.]




\subsection{Algorithm}
[Jump from estimating $\tau_{i,j}$'s to $\tau_i$'s.]

\noindent
We iterate between two main steps: re-center step and re-cluster step.
In the re-center step, we update connecting patterns  based on current clustering  and time shifts; 
while in the re-cluster step, we update the clusters
and time shifts based on updated connecting patterns.

\subsubsection{Re-center step}
% In this step, we estimate connecting patterns  between each pair of clusters $\left\{ f_{q,l} \right\}_{q,l\in[k]}$.
Based on current estimation of clusters $\{  \Gamma_q^{(c)} \}_{q\in[k]}$, we can obtain an (unaligned) point process for each pair of clusters by aggregating connecting times over all pair of nodes in the corresponding two clusters
\begin{align*}
N_{\Gamma_q^{(c)}, \Gamma_l^{(c)}}(\cdot) = \sum_{i\in\Gamma_q^{(c)}, j\in\Gamma_l^{(c)}} N_{i,j}(\cdot).
\end{align*}
Incorporating estimated time shifts $\{  \tau_i^{(c)} \}_{i\in [n]}$ yields two possible alignment: 
(a) $\tilde N_{\Gamma_q^{(c)}, \Gamma_l^{(c)}}(\cdot) = \sum_{i\in\Gamma_q^{(c)}, j\in\Gamma_l^{(c)}} N_{i,j}(\cdot+\tau_i^{(c)})$ 
and (b) $\tilde N_{\Gamma_q^{(c)}, \Gamma_l^{(c)}}(\cdot) = \sum_{i\in\Gamma_q^{(c)}, j\in\Gamma_l^{(c)}} N_{i,j}(\cdot+\tau_j^{(c)})$.
We use the variance of the shifted events as a measure of goodness-of-alignment --- the one producing the smaller variance is adopted.
To be more specific, let 
\begin{align*}
\text{var}_1 &= \text{var}\left[\{ T_{i,j}-\tau_i^{(c)} \}_{i\in\Gamma_q^{(c)}, j\in\Gamma_l^{(c)}}\right],
\\
\text{var}_2 &= \text{var}\left[\{ T_{i,j}-\tau_j^{(c)} \}_{i\in\Gamma_q^{(c)}, j\in\Gamma_l^{(c)}}\right],
\end{align*}
where 
$T_{i,j}$ denotes the edging time between node $i$ and $j$ ($T_{i,j}=\infty$ if there is no edge).
Then
\begin{align}
f_{q,l}^{(u)} &= 
\begin{cases}
\text{density}\left(\{T_{i,j}-\tau_i^{(c)} \}_{i\in\Gamma_q^{(c)}, j\in\Gamma_l^{(c)}}\right), \text{ if}~ \text{var}_1 < \text{var}_2,
\\
\text{density}\left(\{T_{i,j}-\tau_j^{(c)} \}_{i\in\Gamma_q^{(c)}, j\in\Gamma_l^{(c)}}\right), \text{ otherwise}.
\end{cases}
\label{eq: update connecting patterns}
\end{align}
\noindent
[Add how to deal with negative event time after alignment.]



\subsubsection{Re-cluster step}
% In this step we use the updated connecting patterns $\{f_{q,l}^{(u)}\}_{q,l\in[k]}$ to update clusters $\{\Gamma_q\}_{q\in[k]}$ and $\left\{  \tau_i \right\}_{i\in [n]}$.
Given $\{f_{q,l}^{(u)}\}_{q,l\in[k]}$ and $\{\Gamma_q^{(c)}\}_{q\in[k]}$, we propose to solve a simplified version of the objective function in \eqref{eq: objective function 2}
\begin{align}
	\min_{\substack{\{\Gamma_q\}_{q\in[k]},\\  \{\tau_{i,j}\}_{i,j\in[n]}}} 
	\sum_{q,l\in[k]}
	w_{q,l}\cdot
	 d^2(\tilde N_{\Gamma_q, \Gamma_l^{(c)}}, f_{q,l}^{(u)}).
	 \label{eq: simplified obj func}
\end{align}
Setting the weights to be $w_{q,l}\propto \text{deg}(\Gamma_q,\Gamma_l^{(c)}) $, we derive an upper bound of \eqref{eq: simplified obj func} 
\begin{align*}
	\sum_{q,l\in[k]}
	% \frac{\text{deg}(\Gamma_q,\Gamma_l)}{\sum_{q,l}\text{deg}(\Gamma_q,\Gamma_l)}
	&\text{deg}(\Gamma_q,\Gamma_l^{(c)})
	\cdot
	 d^2(\tilde N_{\Gamma_q, \Gamma_l^{(c)}}, f_{q,l}^{(u)})
	 \\
	&\leq
	\sum_{q,l\in[k]}
	% \frac{\text{deg}(\Gamma_q,\Gamma_l)}{\sum_{q,l}\text{deg}(\Gamma_q,\Gamma_l)}
	\text{deg}(\Gamma_q,\Gamma_l^{(c)})
	\cdot
	\left[ \sum_{i\in\Gamma_q}\frac{\text{deg}(i,\Gamma_l^{(c)})}{\text{deg}(\Gamma_q,\Gamma_l^{(c)})} \cdot
	 d^2 \Big( \tilde N_{i, \Gamma_l^{(c)}}, f_{q,l}^{(u)} \Big)  \right] 
	 \\
	 &= \sum_{q}\sum_{i\in\Gamma_q} 
	 \left[ \sum_{l} \text{deg}(i, \Gamma_l^{(c)}) \cdot d^2 \Big( \tilde N_{i, \Gamma_l^{(c)}}, f_{q,l}^{(u)} \Big) \right] 
	 \\
	 &\leq \sum_i \left[ \sum_{l} \sqrt{\text{deg}(i, \Gamma_l^{(c)})} \cdot d \Big( \tilde N_{i, \Gamma_l^{(c)}}, f_{z_i,l}^{(u)} \Big) \right]^2.
	 % \label{eq: simplified obj func}
\end{align*}
The first inequality is obtained by applying Jensen's inequality.
\\
This upper bound leads to our re-cluster step
\begin{align}
z_i^{(u)} 
=
\arg\min_{q\in[k], } \sum_{l} \sqrt{\text{deg}(i, \Gamma_l^{(c)})} \cdot d \Big( \tilde N_{i, \Gamma_l^{(c)}}, f_{q,l}^{(u)} \Big).
\label{eq: update clusters}
\end{align}


\subsubsection*{Update time shifts}
Based on current clusters, we are able to get the unaligned connecting point process $N_{i,\Gamma_q^{(u)}}$ between each node $i$ and each cluster $\Gamma_q^{(u)}$.
We will align nodes from the same cluster and set the minimum time shifts in each cluster as zero for identifiability reason.

For each cluster $\Gamma_q^{(u)}$, randomly select a node $i^*\in\Gamma_q^{(u)}$ as a point of reference. 
The time lag between node $i\in\Gamma_q^{(u)}$ and $i^*$ is then determined by 
\begin{align}
\tau_{(i,i^*),l} &=  \arg\min_{\tau} d(N_{i,\Gamma_l^{(u)}}(\cdot+\tau), N_{i^*,\Gamma_l^{(u)}}(\cdot)) , \quad l\in[k],
\\
\tau_{i,i^*} &= \max \Big( \big|\max_{l\in[k], l\neq q} \tau_{(i,i^*),l}\big|, \big|\min_{l\in[k], l\neq q} \tau_{(i,i^*),l}\big| \Big).
\label{eq: align i and i^*}
\end{align}
\noindent [Add explanation.]
\\
Finally, the time shifts are updated as
\begin{align}
\tau_{i}^{(u)} = \tau_{i,i^*} - \min_{i\in\Gamma_q^{(u)}} \tau_{i,i^*}, \quad i\in\Gamma_q^{(u)}.
\label{eq: update time shifts}
\end{align}











\subsection{Initialization}

Naive k-means algorithm is applied to obtain initialization of clusterings and time shifts.
Time shifts are obtained by aligning each aggregated point process with the first point process, followed by shifting all nodes together so that the minimum time shift is zero.
Clustering 
$\{  \Gamma_q^{(c)} \}_{q\in[k]}$ is then initialized by applying k-means++ algorithm to 
\begin{align*}
\begin{bmatrix}
T_{1,\cdot}-\tau_1^{(c)}\\
T_{2,\cdot}-\tau_2^{(c)}\\
\vdots\\
T_{n,\cdot}-\tau_n^{(c)}
\end{bmatrix}.
\end{align*}




\subsection{Summary of algorithm}

The algorithm is summarized as following. 

\begin{algorithm}[H]
\SetAlgoLined
 Initialize $\{  \Gamma_q^{(c)} \}_{q\in[k]}$ and $\{  \tau_i^{(c)} \}_{i\in [n]}$\;
\While{$\{\Gamma_q^{(c)}\}_{q\in[k]} \neq \{\Gamma_q^{(u)}\}_{q\in[k]}$}{
  	Update $\{f_{q,l}^{(u)}\}_{q,l\in[k]}$ via \eqref{eq: update connecting patterns} with $\{\Gamma_q^{(c)}\}_{q\in[k]}$ and $\{\tau_i^{(c)}\}_{i\in[n]}$ 
  	\;
  	Update $\{\Gamma_q^{(u)}\}_{q\in[k]}$ via \eqref{eq: update clusters} with $\{f_{q,l}^{(u)}\}_{q,l\in[k]}$, $\{\Gamma_q^{(c)}\}_{q\in[k]}$ and $\{\tau_i^{(c)}\}_{i\in[n]}$ 
  	\;
  	Update $\{\tau_i^{(u)}\}_{i\in[n]}$ via \eqref{eq: update time shifts} with $\{\Gamma_q^{(u)}\}_{q\in[k]}$
  	\;
  	Evaluate the stopping criterion
  	\;
  	$\{f_{q,l}^{(c)}\}_{q,l\in[k]} \leftarrow \{f_{q,l}^{(u)}\}_{q,l\in[k]} $
  	\;
  	$\{\Gamma_q^{(c)}\}_{q\in[k]} \leftarrow \{\Gamma_q^{(u)}\}_{q\in[k]} $
  	\;
  	$\{\tau_i^{(c)}\}_{i\in[n]} \leftarrow \{\tau_i^{(u)}\}_{i\in[n]} $
  	\;
 }

\KwOut{$\{f_{q,l}^{(c)}\}_{q,l\in[k]}$, $\{\Gamma_q^{(c)}\}_{q\in[k]}$, $\{\tau_i^{(c)}\}_{i\in[n]} $.}
 \caption{ [name of algorithm]}
\end{algorithm}





		
		% \subsubsection*{k-means in point processes}
		% 	By Assumption \ref{asp:time lag} and \ref{asp:same distr}, 
		% 	% $\lambda_{N_i}(t)=\lambda_{z_i}(t-\tau_i)+\sum_{l=1}^kf_{z_i,l}(t-\tau_i)\epsilon_{i,l}$, hence 
		% 	$\lambda_{N_i}(t+\tau_i)\overset{d}{=}\lambda_{N_j}(t+\tau_j)$ for any $i,j$ satisfying $z_i=z_j$. Thus $F_{N_i}(t+\tau_i)\overset{d}{=}F_{N_j}(t+\tau_j)$ for any $i,j$ satisfying $z_i=z_j$, 
		% 	where 
		% 	$F_{N_i}(t):=\int_0^t\lambda_{N_i}(s)\text{d}s\big/\int_0^T\lambda_{N_i}(s)\text{d}s$.
		% 	So the k-means problem is to solve
		% 	% \begin{equation}\label{eq:kmeans_lambda}
		% 	% \min_{\left\{ \Gamma_l \right\}_{l=1}^k} \frac{1}{n} \sum_{l=1}^k \left(\min_{\{\tau_i\}_{i\in\Gamma_l},\lambda_l} \sum_{i\in\Gamma_l} \|\hat\lambda_{N_i}(\cdot+\tau_i)- \lambda_l(\cdot)\|_2^2 \right),
		% 	% \end{equation}
		% 	% or, equivalently,
		% 	\begin{align}\label{eq:kmeans_F}
		% 	\min_{\left\{ \Gamma_l \right\}_{l=1}^k} \frac{1}{n} \sum_{l=1}^k \left(\min_{\{\tau_i\}_{i\in\Gamma_l},F_l} \sum_{i\in\Gamma_l} \| \tilde{F}_{N_i}(\cdot+\tau_i)- F_l(\cdot)\|_2^2 \right),
		% 	\end{align}
		% 	where 
		% 	% $\hat \lambda_{N_i}(\cdot)$ is the intensity function estimated from $N_i(\cdot)$ using some smooth method,
		% 	$\tilde{F}_{N_i}(t):=1/{N_i([0,T])}\cdot\sum_{j=1}^{N_i([0,T])}\mathbf{1}_{\{t_{N_i,j}\leq t\}}$ is the empirical distribution function of the occurrence time of edges,
		% 	$t_{N_i,j}$ is the occurrence time of the $j$-th edge of $N_i(\cdot)$,
		% 	and $F_l(t)=\mathbb{E}[F_{N_i}(t+\tau_i)] (\forall i\in\Gamma_l)$ is the expected cumulative distribution function of the $l$-th cluster.


		% 	% Likelihood can also be used as a measure of similarity between a point process and an intensify function,
		% 	% which yields the objective function
		% 	% \begin{align}\label{eq:kmeans_likelihood}
		% 	% \min_{\left\{ \Gamma_l \right\}_{l=1}^k} \frac{1}{n} \sum_{l=1}^k \left(\min_{\{\tau_i\}_{i\in\Gamma_l},\lambda_l} \sum_{i\in\Gamma_l} d\big({N_i}(\cdot+\tau_i), \lambda_l(\cdot)\big) \right),
		% 	% \end{align}
		% 	% where the distance is defined as the negative log-likelihood
		% 	% \begin{align*}
		% 	% d(N, \lambda) := -\log L(N;\lambda) = \int_{0}^T\lambda(t)\text{d}t - \sum_{j=1}^{N(0,T]}\log \left( \lambda(t_{N,j}) \right) .
		% 	% \end{align*}
				
		% 	% {\color{blue} Justify why it is reasonable to use Poisson process. }
		% 	% % Also explain why this is a good measure of similarity.
	 % 		% {\color{blue} See \citet{Daley} for details.}

	 % 	% \subsubsection*{Other possible distance}
		% 	% The squared error distance is defined as 
		% 	% \begin{align*}
		% 	% d(N,\lambda) := \int_0^T\lambda^2(t)\text{d} t - 2\int_0^T \lambda(t)\text{d}N(t)+?
		% 	% \end{align*}



	% \subsection{Algorithm}
		
	% 	The initialization method and the choice of the number of clusters $k$ will be discussed later, for now we assume the initialization and $k$ are given.
	% 	To solve the problem \eqref{eq:kmeans_F}, we iterate between two steps until convergence:
	% 		\begin{itemize}
	% 			\item Re-cluster step: update the clustering 
	% 			$\left\{ \hat\Gamma_l \right\}_{l=1}^k$ based on the distance $d(\tilde{F}_{N_i}, \hat F_l)$ 
	% 			defined as 
	% 			\begin{align*}
	% 			 d(\tilde{F}_{N_i}, \hat F_l) = \min\Big\{ \inf_{\tau\in[0,T]}\left( \int_{-T}^T\big| S_\tau\circ\tilde{F}^*_{N_i}(t)-\hat F_{l}^*(t) \big|^2 \text{d}t \right)^{1/2}, \\
	% 			 \inf_{\tau\in[0,T]}\left( \int_{-T}^T\big| \tilde{F}^*_{N_i}(t)-S_\tau\circ\hat F_{l}^*(t) \big|^2 \text{d}t \right)^{1/2} \Big\},
	% 			\end{align*}
	% 			where
	% 			\begin{align}\label{eq:def of shifted curve}
	% 			S_\tau\circ\tilde{F}^*_{N_i}(t)=
	% 			\begin{cases}
	% 			0, &t\in[-T,-\tau)\\
	% 			\tilde F_{N_i}(t+ \tau), &t\in[-\tau,T-\tau)\\
	% 			1,& t\in [T-\tau,T]
	% 			\end{cases},
	% 			% \quad
	% 			\hat F_{l}^*(t)=
	% 			\begin{cases}
	% 			0, &t\in[-T,0)\\
	% 			\hat F_{l}(t), &t\in [0,T]
	% 			\end{cases}.
	% 			\end{align}
				
				
	% 			\item Re-center step: update the expected cumulative distribution functions $\{\hat F_l\}_{l=1}^k$ using the method in \citet{Bigot2013}.

	% 		\end{itemize}

	% 	In the re-cluster step, the distance $d(\tilde F_{N_i}, \hat F_l)$ for each pair of node $i$ and cluster $l$ is evaluated by solving the problem
	% 	\begin{equation}
	% 	\begin{aligned}\label{eq:aligned distance}
	% 	\hat n_{i,l}&=
	% 	\underset{|n|\leq (N-1)/2}{\arg\min}
	% 	\sum_{0<|j|\leq (N-1)/2}
	% 	\left| \theta_{i,j}e^{\complexunit 2\pi j n/N}+
	% 	\frac{e^{\complexunit{}2\pi j n/N}-1}{1-e^{\complexunit{}2\pi j /N}}
	% 	-\gamma_{l,j} \right|^2
	% 	+ \left| \theta_0+n-\gamma_0 \right|^2
	% 	\\
	% 	&=\underset{|n|\leq (N-1)/2}{\arg\min}
	% 	\sum_{0<|j|\leq (N-1)/2}
	% 	\left| \left( \theta_{i,j}+ \frac{1}{1-e^{\complexunit{}2\pi j /N}} \right) e^{\complexunit{}2\pi j n/N} 
	% 	- \left( \gamma_{l,j}+\frac{1}{1-e^{\complexunit{}2\pi j /N}} \right)  \right|^2 
	% 	+\\
	% 	&\hspace{11cm} \left| \theta_0+n-\gamma_0 \right|^2
	% 	\\
	% 	&\overset{\triangle}{=}
	% 	\underset{|n|\leq (N-1)/2}{\arg\min}\sum_{0<|j|\leq (N-1)/2}\left| \theta_{i,j}' e^{\complexunit{}2\pi j n/N} - {\gamma_{l,j}'}  \right|^2
	% 	+ \left| \theta_0+n-\gamma_0 \right|^2
	% 	,
	% 	\end{aligned}
	% 	\end{equation}
	% 	where $n=N\cdot\tau/2T$,
	% 	$\theta_{i,j}$ and $\gamma_{l,j}$,
	% 	$j=  -(N-1)/2,\cdots,(N-1)/2 $, are the discrete Fourier coefficients of $\tilde F^*_{N_i}$ and $\hat F^*_l$,
	% 	$N$ is the length of discretized version of $\tilde F^*_{N_i}$ and $\hat F^*_l$.
	% 	% Grid search can be used to solve problem \eqref{eq:aligned distance}.
	% 	\\
	% 	Gradient descent can be used to solve the problem. 
	% 	The gradient is shown below:
	% 	% Gradient:
	% 	\begin{equation}
	% 	\begin{split}
	% 	\nabla_{n_i} = \frac{4\pi}{N}\cdot 
	% 	\sum_{0<|j|\leq (N-1)/2} j\cdot 
	% 	\text{Im} \left( {\theta_{i,j}'}\overline{{\gamma_{i,j}'}}e^{\complexunit2\pi n_i j/N} \right) 
	% 	+ 2n_0+2 \text{Re}\left( \theta_0-\gamma_0 \right) .
	% 	\end{split}
	% 	\end{equation}
	% 	The learning rate was set as $0.01$, the initialization of $n_i$ was set to be the last estimated $\hat n_i$.
		


	% 	In the re-center step, let $\left\{ \theta_{i,j} \right\}_{j\in \mathbb{Z}}$ and $S_\tau\circ\tilde{F}^*_{N_i}(t)$ be defined the same as above.
	% 	The Fourier coefficients of $S_\tau\circ\tilde{F}^*_{N_i}(t)$ are
	% 	\begin{align*}
	% 	\theta_{i,j}e^{\complexunit 2\pi j n/N}+
	% 	\frac{1}{1-e^{\complexunit{}2\pi j /N}}
	% 	\left( e^{\complexunit{}2\pi j n/N}-1 \right) 
	% 	\overset{\triangle}{=}
	% 	\theta_{i,j}' e^{\complexunit{}2\pi j n/N}-C. 
	% 	\end{align*}
	% 	Then
	% 	$\left\{ \hat \tau_i \right\}_{i=1}^n = \left\{ \hat n_i\cdot 2T/N \right\}_{i=1}^n$ can be obtained by 
	% 	\begin{align*}
	% 	\left\{ \hat n_i \right\}_{i\in\Gamma_l} &= 
	% 	\underset{\min_{i\in\Gamma_l} \left\{ n_i \right\}=0}{\arg\min}
	% 	\frac{1}{|\Gamma_l|}
	% 	\sum_{i\in\Gamma_l}
	% 	\sum_{j=1}^N 
	% 	\left| 
	% 	\theta_{i,j}' e^{\complexunit{}2\pi j n_i/N} -
	% 	\frac{1}{|\Gamma_l|}\sum_{i'\in\Gamma_l}\theta_{i',j}' e^{\complexunit{}2\pi j n_{i'}/N}  
	% 	\right|^2 
	% 	.
	% 	\end{align*}
	% 	Using gradient descent over all $\left\{ n_i \right\}_{i\in\Gamma_l}$ is expensive, thus we adopt the following two-step estimation procedure:
	% 	\begin{itemize}
	% 		\item Estimate the mean distribution function $\hat F^*_l(t)$.
	% 		\item Estimate $\left\{ \hat n_i \right\}_{i\in\Gamma_l}$ by aligning each $\tilde F^*_{N_i}(t) $ with $\hat F^*_l(t)$.
	% 	\end{itemize}
	% 	The initialization of $\hat F^*_l(t)$ can be any randomly chosen $\tilde F^*_{N_i}(t) $.

	% 	% gradient:
	% 	% \begin{align*}
	% 	%  \frac{\partial L}{\partial \tau_i} = 
	% 	%  \frac{1}{|\Gamma_l|^2}
	% 	%  \sum_{|j|\leq m} 4\pi j
	% 	%  \left( 
	% 	%  	\text{Im}\left( \theta_{i,j}' e^{\complexunit{}2\pi j \tau_i}\cdot\overline{A_j}\right)+
	% 	%  	\left(\frac{1}{|\Gamma_l|}-2\right) \text{Im}(A_j)\cdot 2 \text{Re}\left( \theta_{i,j}' e^{\complexunit{}2\pi j \tau_i}   \right) 
	% 	%  \right) ,
	% 	%  \end{align*}
	% 	%  where $A_j=\frac{1}{|\Gamma_l|}\sum_{i\in\Gamma_l}\theta_{i,j}'e^{\complexunit{}2\pi j \tau_i} $.

	% 	Finally, the mean distribution function is estimated as the average of shifted empirical distribution functions.		  



		% {\color{red} Should we take into account the error in $t_{N_i,j}$?}\\
		


		
		% Another direction is to use negative log-likelihood as the distance.
		% For reference see \cite{Vimond2010,Gamboa2007,Ronn2009,Gervini2005}.
		% The maximum likelihood estimator of the mean curve proposed in \citet{Gervini2005} is showed to be $\sqrt{n}$-consistent and asymptotically normal. % {(converge to a Gaussian process)}
		% {\color {red} But the log-likelihood includes unknown $f_{l,l'}$}
		% \begin{align*}
		% L(\lambda_{l},\tau_i;N_i)=\int \mathbb{P}\left( N_i(t)\Big| \lambda_{l}(t-\tau_i)+\sum_{l'=1}^k f_{l,l'}(t-\tau_i)\epsilon_{i,l'} \right) f(\epsilon_{i,1})\cdots f(\epsilon_{i,k})\\\text{d}\epsilon_{i,1}\cdots \text{d}\epsilon_{i,k}
		% \end{align*}
			


	% \subsection{Convex relaxation of k-means type clustering}
	% 	\subsubsection*{Semidefinite programming relaxation}
	% 		We briefly introduce a semidefinite programming relaxation (Peng-Wei relaxation) of k-means proposed by \citet{Peng2005}.
	% 		The k-means objective function in \eqref{eq:kmeans} can be re-written as
	% 		\begin{align*}
	% 		\sum_{l=1}^k\sum_{i\in\Gamma_l}\|\mathbf{x}_i- \mathbf{c}_l\|^2 &= \frac{1}{2}\sum_{l=1}^k \frac{1}{|\Gamma_l|}\sum_{i,j\in\Gamma_l}\|\mathbf{x}_i - \mathbf{x}_j\|^2\\
	% 		&= \frac{1}{2}\sum_{l=1}^k\frac{1}{|\Gamma_l|}\langle \mathbf{1}_{\Gamma_l}\mathbf{1}_{\Gamma_l}^\top,\mathbf{D} \rangle
	% 		\end{align*}
	% 		where $\mathbf{D}\in \mathbb{R}^{n\times n}$ with entries $\mathbf{D}_{ij}=\|\mathbf{x}_i- \mathbf{x}_j\|^2$.
	% 		\\
	% 		Hence \eqref{eq:kmeans} can be relaxed to
	% 		\begin{align*}
	% 		&\min_{\mathbf{Z}}\ \langle \mathbf{Z},\mathbf{D}\rangle \qquad \\
	% 		& \ \text{s.t. } \quad \mathbf{Z} \succeq 0, \quad \mathbf{Z} \geq 0, \quad \mathbf{Z} 1_{n}=\mathbf{1}_{n}, \quad \operatorname{Tr}(\mathbf{Z})=k.
	% 		\end{align*}
	% 	Proximity conditions are discussed in \ref{sec:proximity condition}.
		
	% 	In our case, \eqref{eq:kmeans_lambda} is equivalent to 
	% 		\begin{equation}\label{eq:unconvexified k-means}
	% 		\min_{\mathbf{Z},\left\{ \tau_i \right\}_{i=1}^n}\langle \mathbf{Z}, \mathbf D(\tau_1,\cdots,\tau_n)  \rangle
	% 		\end{equation}
	% 	where $\mathbf{Z}=1/2\cdot \sum_{l=1}^k(1/|\Gamma_l|\cdot \mathbf{1}_{\Gamma_l}\mathbf{1}_{\Gamma_l}^\top)$, $\mathbf{D}_{i,j} = \| \hat\lambda_{N_i}(t+\tau_i)-\hat\lambda_{N_j}(t+\tau_j) \|_2^2$.
	% 	Using Fourier transformation, 
	% 	\begin{align*}
	% 	\mathbf{D}_{i,j} &= \int_{0}^T \left( \int_0^\infty \hat h_{N_i}(\xi)e^{i2\pi\xi(t+\tau_i)}-\hat h_{N_j}(\xi)e^{i2\pi\xi(t+\tau_j)}\text{d}\xi \right)^2 \text{d}t
	% 	\end{align*}
	% 	where $\hat h_{N_i}(\xi)$ is the Fourier transform of $\hat\lambda_{N_i}(t)$. 
	% 	% This seems not a convex function of $\tau_i$'s.
	% 	{\color{red} How to convexify?}
		
		
		

	% \subsection{Other thoughts}
	% 	\begin{itemize}
	% 		\item What about using Fourier transformation and then k-means? What about functional principal component analysis?
	% 	\end{itemize}








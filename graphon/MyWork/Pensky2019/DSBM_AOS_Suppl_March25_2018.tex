%  This is the file with proofs
% The root file is DSBM_AOS_March25_2018.tex


\section{Supplemental Material:\ Proofs}
\label{sec:suppl}
% \setcounter{equation}{0}


\renewcommand{\theequation}{S.\arabic{equation}}
 \setcounter{equation}{0}


%%%%%%%%%%%%%%%%%%%%%%%%%%%%%%%%%%%%%%%%%%%%%%%%%%%%%%%%%%%%%%%%%%%%%%%%%%%%%%%%%%%%%%%%%%%%%%%%%


 %  This is the file with proofs
% The root file is DSBM_AOS_March25_2018.tex


% \section{Proofs}
% \label{sec:proofs}
%\setcounter{equation}{0}

% \subsection{Proofs of the upper bounds for the risk}
%\label{sec:upper}
\subsection{Proof of Theorem \ref{th:oracle}}

%  \noindent
% {\bf Proof of Theorem \ref{th:oracle}. } 
%
Since $(\hbd, \hbC, \hm, \hJ)$ are solutions of optimization problem \fr{opt_problem},
for any $m$, $J$, $\bC$  and $\bd$ one has
\be \label{main_ineq}
\|\ba - \hbC \hbW^T\, \hbd_{(\hJ)}\|^2 + \Pen(|\hJ|,\hm) \leq \|\ba - \bC \bW^T \bd_{(J)} \|^2 + \Pen(|J|,m),
\ee 
% where $\bd_{(J)}$ is the modification of vector 
% $\bd$ where all elements $\bd_j$ with $j \notin J$ are set to zero.
For any $m$, $J$, $\bC$  and $\bd$, it follows from \fr{true_model} that 
\begin{align*} 
& \|\ba -   \bC \bW^T  \bd_{(J)}\|^2  = \|(\ba - \bCs \bWs^T \bds) +  (\bCs \bWs^T \bds -  \bC \bW^T \bd_{(J)})\|^2 \\
& =   \|\bxi\|^2 + \|\bC \bW^T \bd_{(J)} - \bCs \bWs^T \bds\|^2 + 2 \bxi^T (\bCs \bWs^T \bds -  \bC \bW^T \bd_{(J)}).
\end{align*}
Hence, plugging the last identity into the inequality \fr{main_ineq}, derive that
for any $m$, $J$, $\bC$  and $\bd$
\begin{align} \label{main_ineq1}
\|\hbC \hbW^T \hbd_{(\hJ)} - \bCs \bWs^T \bds\|^2 & \leq \|\bC \bW^T \bd_{(J)} - \bCs \bWs^T \bds\|^2 \\
& +    \Del + \Pen(|J|,m) - \Pen(|\hJ|,\hm). \nonumber 
\end{align}
%
Here $\Del   = 2 |\bxi^T (\hbC \hbW^T \hbd_{(\hJ)} - \bC \bW^T \bd_{(J)})| \leq  \Del_1 + \Del_2 + \Del_3$ with 
\begin{align}
 % \Del & = 2 |\bxi^T (\hbC \hbW^T \hbd_{(\hJ)} - \bC \bW^T \bd_{(J)})| \leq  \Del_1 + \Del_2 + \Del_3, \label{Del_sum}\\
  \Del_1 & = 2 |\bxi^T (\bCs \bWs^T \bds - \bC \bW^T \bd_{(J)})|, \nonumber \\
  \Del_2  & = 2 |\bxi^T (\bI_{NL} - \hPhJ) \bCs \bWs^T \bds|, \quad 
  \Del_3   = 2  \bxi^T \hPhJ\, \bxi,  \label{Del123}
\end{align}
% \be \label{Del_sum}
% \Del   = 2 |\bxi^T (\hbC \hbW^T \hbd_{(\hJ)} - \bC \bW^T \bd_{(J)})| \leq  \Del_1 + \Del_2 + \Del_3,
% \ee
since, due to \fr{projection} and \fr{hbd_solution}, one has  $\hbC \hbW^T \hbd_{(\hJ)} = \hPhJ\, \ba$ 
where $\ba$ is given by \fr{true_model}. Now, we need to find upper bounds for each of the terms in \fr{Del123}.
 


By Lemma \ref{lem:bern_err} with $\al =1/2$ and any   $t>0$, one has
\be \label{Del1er}
\PP \lfi \Del_1 -  0.5\, \| \bCs \bWs^T \bds - \bC \bW^T \bd_{(J)} \|^2 \leq 4 t  \rfi \geq 1 - 2 e^{-t}.
\ee
% 
Note that  
\begin{align*}
\|\hbC \hbW^T \hbd_{(\hJ)} - \bCs \bWs^T \bds\|^2 & = 
\|\hPhJ\, \bxi\|^2 + \| (\bI_{NL} - \hPhJ)  \bCs \bWs^T \bds\|^2 \\
& \geq  \| (\bI_{NL} - \hPhJ)  \bCs \bWs^T \bds\|^2.
\end{align*}
Therefore,  applying an union bound over $m=1, \cdots, n$, $\bC \in \calC(m,n,L)$ 
and $J$ with $|J| = 1, \cdots, ML,$ we derive that for    any $x >0$ 
\begin{align*} 
&  \PP \lfi \Del_2 -  0.5\, \|\hbC \hbW^T \hbd_{(\hJ)} - \bCs \bWs^T \bds\|^2 < 4x \rfi   \\  
 \geq & \PP \lfi \Del_2 -    0.5\,\| (\bI_{NL} - \hPhJ)  \bCs \bWs^T \bds\|^2 < 4x \rfi    \\
 \geq & 1 - \sum_{m=1}^n\ \sum_{\bC \in \calC(m,n,L)}\ \sum_{j =1}^{ML} \ \sum_{|J|=j} 
\PP \lfi 2 |\bxi^T (\bI_{NL} - \hPhJ) \bC  \bW^T \bds|   \right. \\
- & \left.  0.5\,\| (\bI_{NL} - \hPhJ)  \bC  \bW^T \bds\|^2 < 4x \rfi. 
\end{align*}
Denote
\be \label{RDel}
R(m,J,L) =   \log(|\calC(m,n,L)|) + |J| \log\lkr  m^2 L e\,|J|^{-1} \rkr +  2 \log(m |J|). 
\ee
%
Then, taking into account that the number of sets $J$ with $|J|=j$ is
$$
{ ML \choose j} \leq \lkr \frac{M L e}{j}\rkr^j \leq \lkr \frac{m^2 L e}{j}\rkr^j
$$
and applying  Lemma \ref{lem:bern_err} with $\al =1/2$ and $x = t + R(m,J,L)$, derive
\bes
\PP \lfi \Del_2 - 0.5\, \|\hbC \hbW^T \hbd_{(\hJ)} - \bCs \bWs^T \bds\|^2 - 4 R(\hm,\hJ,L)
\leq 4t  \rfi \geq 1 -  2 e^{-t} \, \sum_{m=1}^n\  \sum_{j =1}^{ML} m^{-2} j^{-2}.
\ees
Since $\sum_{j=1}^\infty  j^{-2} = \pi^2/6 < \sqrt{3}$, the last inequality yields
\be \label{Del2er}
\PP \lfi \Del_2 - 0.5\, \|\hbC \hbW^T \hbd_{(\hJ)} - \bCs \bWs^T \bds\|^2 - 4 R(\hm,\hJ,L)
\leq 4t  \rfi \geq 1 - 6 e^{-t}.
\ee
% 
Finally, in order to obtain an upper bound for $\Del_3$, apply Lemma \ref{corr:quadr_err}
with $\bA =  \hPhJ$  and again use the union upper bound over $m=1, \cdots, n$, $\bC \in \calC(m,n,L)$ 
and $J$ with $|J| = 1, \cdots, ML$ similarly to the way it was done for $\Del_2$. 
Since for any projection matrix $\PJ$, one has $\|\PJ\|_{op}=1$ and
$\| \PJ\|^2  = |J|$, obtain that for any $t>0$  
\be \label{Del3er}
\PP \lfi \Del_3 - |\hJ| - \frac{3}{2} R(\hm,\hJ,L)
\leq \frac{3t}{2} \rfi \geq 1 -  e^{-t},
\ee
where $R(m,J,L)$ is defined in \fr{RDel}.
% 
Combining \fr{main_ineq1}--\fr{Del3er} and recalling that $\hbte = \hbC \hbW^T \hbd_{(\hJ)}$ and $\btes = \bCs \bWs^T \bds$, 
obtain  that, with probability at least $1 - 9 e^{-t}$, one has
\be \label{in_eq}
% \PP \lfi \|\hbte - \btes\|^2 \leq   \min_{\stackrel{m,J,\bq}{\bC \in \calC(m,n,L)}} 
% \lkv  3\, \| \bC \bW^T \bd_{(J)} - \btes \|^2  +  2\, \Pen(|J|,m)  \rkv + 19\,  t \rfi \geq 1 - 9 e^{-t} 
% \PP \lfi \|\hbte - \btes\|^2 \leq   \min_{\stackrel{m,J,\bq}{\bC \in \calC(m,n,L)}} 
% \lkv  3\, \| \bC \bW^T \bd_{(J)} - \btes \|^2  +   11  R(m,J,L)  + 2|J|  \rkv + 19\,  t \rfi \geq 1 - 9 e^{-t}. 
\|\hbte - \btes\|^2 \leq   \min_{\stackrel{m,J,\bq}{\bC \in \calC(m,n,L)}} 
\lfi  3  \| \bC \bW^T \bd_{(J)} - \btes \|^2  +   11  R(m,J,L)  + 2|J|  \rfi + 19t. 
\ee 
 In order to complete the proof of \fr{oracle_prob}, observe that 
$2 \log(|J|) \leq 2|J|$ and $\log(|\calC(m,n,L)|\geq  2 \log m$ by \fr{clust_assump}.
Therefore, one has
\be \label{pen_cond}
 11  R(m,J,L)  + 2|J| \leq 2 \Pen(|J|,m),
 \ee
and \fr{oracle_prob} follows from \fr{in_eq}, \fr{pen_cond} and the second inequality in \fr{symrel}.
 
Finally, inequality \fr{oracle_expec} can  be proved by noting that for any random variable $\zeta$ one has
$\EE \zeta \leq \int_0^\infty \PP(\zeta >z) dz$ and using it with $\zeta = \|\hbLam - \bLams\|^2$.
 

 %%%%%%%%%%%%%%%%%%%%%%%%%%%%%%%%%%%%%%%%%%%%%%%%%%%%%%%%%%%%%%%%%%%%%%%%%%%%%%%%%%%%%%%%%%%%%%%%%%%%%%%%%%%%%%%%%%%%
%%%%%%%%%%%%%%%%%%%%%%%%%%%%%%%%%%%%%%%%%%%%%%%%%%%%%%%%%%%%%%%%%%%%%%%%%%%%%%%%%%%%%%%%%%%%%%%%%%%%%%%%%%%%%%%%%%%%


\subsection{ Proofs of Lemma~\ref{lem:balanced},   Theorem~\ref{th:upper_smooth_DSBM} and Corollary~\ref{cor:upper_smooth_DSBM}}
This section contains proofs of Lemma~\ref{lem:balanced},   Theorem~\ref{th:upper_smooth_DSBM} 
and Corollary~\ref{cor:upper_smooth_DSBM}.
\\



%%%%%%%%%%%%%%%%%%%%%%%%%%%%%%%%%%%%%%%%%%%%%%%%%%%%%%%%%%%%%%%%%%%%%%%%%%%%%%%%%%%%%%%%%%%%%%%%%%%%%%%5


\noindent
{\bf Proof of Lemma \ref{lem:balanced}.\ }
% The upper bound in \fr{eq:balanced} is trivial. 
% To prove the lower bound in \fr{eq:balanced},
Since
$|\calZ_{\bal} (m,n,n_0,L,1,1)| \leq |\calZ_{\bal} (m,n,n_0,L,\aleph_1,\aleph_2)|$,
 it is sufficient to prove \fr{card_balanced} for $\aleph_1 = \aleph_2=1$. 
%
Note that 
\bes 
\log|\calZ_{\bal} (m,n,n_0,L,1,1)| \geq  \log(n!) - m \log [(n/m)!] + (L-1) \log{n \choose n_0}
\ees
since there are ${n \choose n_0}$ ways to select $n_0$ nodes out of $n$ but there are more than 
one way to put them back. Applying Lemma \ref{lem:cardinality} with $\ga =1$ obtain that
$\log(n!) - m \log [(n/m)!] \geq n \log(m)/4$. In addition,
\bes
\log{n \choose n_0} > n_0 \log \lkr \frac{n}{n_0}\rkr = \frac{n_0}{4} \log \lkr \frac{n^4}{n_0^4}\rkr
\geq \frac{n_0}{4} \log \lkr \frac{mne}{n_0}\rkr
\ees
provided $n^4/n_0^4 \geq (mne)/n_0$ which holds under conditions of the lemma due to $m \leq n$.
\\

%%%%%%%%%%%%%%%%%%%%%%%%%%%%%%%%%%%%%%%%%%%%%%%%%%%%%%%%%%%%%%%%%%%%%%%%%%%%%%%%%%%%%%%%%%%%%%%%%%%%%%%5


\noindent
{\bf Proof of Theorem~\ref{th:upper_smooth_DSBM}. }
Let  $\ms$ be the true number of classes,  $\Ms = \ms(\ms +1)/2$.
Let $\bC = \bCs$ be the true clustering matrix and $(\bCs)^T \bCs = (\bSs)^2$ where $(\bSs)^2$ is the diagonal 
matrix with the number of nodes in respective pairs of classes on the diagonal. 
Let $\bQs$ be the true matrix of probabilities of connections for pairs of classes, $\bDs = \bQs \bH$, $\bds = \vect(\bDs)$,
$\btes = \bCs (\bWs)^T \bds$ and $\bWs = \bH \otimes \bI_{\Ms}$. 
We need to find an upper bound for  $\| \bCs  (\bWs)^T \bds_{(J)} - \btes \|^2$ in \fr{oracle_prob}. 
Let $\bQs_{(J)}$ be such that 
\bes
 \vect(\bQs_{(J)}) =  \bqs_{(J)} = (\bWs)^T \bds_{(J)} =  (\bWs)^T (\bWs \bqs)_{(J)}. %, \bU = \bC^T \bS^{-1}.
\ees
Then, by direct calculations, one obtains  
\begin{align*}
\| \bCs  (\bWs)^T \bds_{(J)} - \btes \|^2 & = \lkv \bqs_{(J)}  - (\bSs)^{-2} \bCs \btes\rkv^T  (\bSs)^2  \lkv \bqs_{(J)} - (\bSs)^{-2} \bCs \btes\rkv \\
& +  \|\btes\|^2 - \|(\bSs)^{-1} (\bCs)^T \btes \|^2.
\end{align*}
Since  $\|(\bSs)^{-1} (\bCs)^T \btes\| = \|\btes\|$ and $(\bSs)^{-2} \bCs \btes = \bqs = \vect(\bQs) = \vect(\bDs \bH)$,
% where $\bQs$ and $\bDs$  are the true values of matrices $\bQ$ and $\bD$. 
% Moreover, $\tilbq_J  = \bW^T \bd_{(J)} = \vect(\bQ_J)$.
obtain 
\begin{align*}
\| \bCs  (\bWs)^T \bds_{(J)} - \btes \|^2 & =  \|\bSs (\bQs_{(J)} - \bQs)\|^2 \\
& \leq  \sum_{k=1}^{\Ms} \max_{l}  N_{k}^{(l)}\ \sum_{l=1}^L \lkv (\bQs_{(J)})_{k,l} - \bQs_{k,l} \rkv^2\\
& \leq \aleph_2^2 \lkr \frac{n}{\ms} \rkr^2 \, \sum_{k=1}^{\Ms} \|\bQs_{(J)})_{k,*} - \bQs_{k,*}\|^2 
\end{align*}
Here, $k$ is the index corresponding to a pair of classes $(k_1, k_2)$,   $N_{k_1, k_2}^{(l)}$ is 
defined in formula \fr{eq:Nkl} and the second inequality follows from assumption \fr{balanced}.
In order to complete the proof, note that 
\begin{align*}  
% \sum_{k=1}^{\Ms} \|\bQ_{(J)})_{k,*} - \bQs_{k,*}\|^2 & = 
\sum_{k=1}^{\Ms} \|\bQs_{(J)})_{k,*} - \bQs_{k,*}\|^2 & = \sum_{k=1}^{\Ms}   \|\bDs_{(J_k)})_{k,*} - \bDs_{k,*}\|^2 
% \\ &   
= \sum_{k=1}^{\Ms}   \sum_{l \notin J_k} (\bDs_{k,l})^2.   
\end{align*} 
Therefore, \fr{oracle_prob} implies \fr{oracle_prob_specific}. 
\\



%%%%%%%%%%%%%%%%%%%%%%%%%%%%%%%%%%%%%%%%%%%%%%%%%%%%%%%%%%%%%%%%%%%%

\noindent
{\bf Proof of Corollary~\ref{cor:upper_smooth_DSBM}.}
Observe that it follows from \fr{bias_coef_cond} that one can choose 
$J_k = \{l:\ 1 \leq l \leq L_0\}$ where $L_0 \leq L$. Then, 
\fr{eq:J_union} yields that $|J| = L_0 \Ms$.
Moreover, due to assumption \fr{bias_coef_cond}, obtain 
\bes 
\sum_{k=1}^{\Ms}\, \sum_{l \notin J_k} (\bDs_{k,l})^2 \leq K_0 \Ms L_0^{-2 \nu_0}.
\ees
Note also that if $L_0 = L$, then there is no bias and the sum in \fr{oracle_prob_specific} is identical zero. 
Then, \fr{oracle_prob_specific} becomes
\begin{align*}   
 \|\hbLam - \bLams\|^2  & \leq \min \lfi \Del(n,L,\ms),\ 22 \Ms L   \rfi\\
& + 44 \lkv n \log (\ms)  + 
   n_0 L  \,\log\lkr \frac{\ms ne}{n_0}\rkr \rkv + 38\, t,
\end{align*}
where
\bes % \label{eq:Del_expression}
\Del(n,L,\ms) = \min_{0 \leq L_0 <L} \lfi 6  {K}_0  \aleph_2^2  n^2\, L_0^{-2 \nu_0} 
+ 22 (\ms)^2 L_0 \log\lkr \frac{25\,L}{L_0}\rkr \rfi
\ees 
In order to obtain \fr{oracle_prob_smooth}, minimize the right-hand side of the last expression 
with respect to $L_0<L$ and note that, if $L_0 <L$, then
$\log(L_0) \asymp \log (L) \asymp \log(n/\ms)$.   



 
%%%%%%%%%%%%%%%%%%%%%%%%%%%%%%%%%%%%%%%%%%%%%%%%%%%%%%%%%%%%%%%%%%%%%%%%%%%%%%%%%%%%%%%%%%%%%%%%%%%%%%%%%%%%%%%%%%%%
%%%%%%%%%%%%%%%%%%%%%%%%%%%%%%%%%%%%%%%%%%%%%%%%%%%%%%%%%%%%%%%%%%%%%%%%%%%%%%%%%%%%%%%%%%%%%%%%%%%%%%%%%%%%%%%%%%%%



\subsection{ Proofs of Theorems~\ref{th:DSBM_lower_bounds} and \ref{th:smooth_DSBM_lower_bounds}}
%\label{sec:lower}


This section contains the proofs of the lower bounds for the error.
The lower bounds in both, Theorem \ref{th:DSBM_lower_bounds} and Theorem~\ref{th:smooth_DSBM_lower_bounds},
consist  of two parts, the clustering error and the nonparametric estimation error.
We shall consider those terms separately.
\\  

%%%%%%%%%%%%%%%%%%%%%%%%%%%%%%%%%%%%%%%%%%%%%%%%%%%%%%%%%%%%%%%%%%%%%%%%%%%%%%%%%%%%%%%%%%%%%%%%%%%%%%%%%%%%%%%%%%%%


\noindent
{\bf Proof of Theorem \ref{th:DSBM_lower_bounds}. }
Although the upper bounds for the risk in Corollary~\ref{cor:upper_DSBM}
are derived for the case of general clustering matrices, due to the fact that the balanced model clustering complexity
is the same as complexity of general clustering, we derive the lower bounds for the clustering error for the case 
when $\bC \in  \calZ_{\bal} (m,n,n_0,L,\aleph_1, \aleph_2)$. Moreover, since the case of $\aleph_1 = \aleph_2 =1$
is the most restrictive, we prove the clustering error for this case.
\\
 

\underline{\bf The clustering error. } 
Without loss of generality, assume that   $\ga m$ and $\ga n$ are integers.  
Assume that connectivity tensor $\bG$ does not change with $l$, so   $\bG_{*,*,l} = \bV$ is an $m\times m$ symmetric matrix.
Let $\bV$ be block diagonal and  such that the diagonal blocks are equal to zero
and the non-diagonal blocks are equal to $\bF$ and $\bF^T$, respectively, so that
$\bV_{k_1,k_2} =0$ if $1 \leq k_1,k_2 \leq (1-\ga)m$ or 
$(1-\ga)m+1 \leq k_1,k_2 \leq m$ and  
$\bV_{k_1,(1-\ga)m +k_2}  = \bF_{k_1,k_2}$ if  $k_1 = 1 ,\cdots (1-\ga)m$, $k_2= (1-\ga)m+1, \cdots,   m$. 
Since components of vectors $\bG_{k_1, k_2, *}$ are constant for any $k_1$, $k_2$, then,
due to condition \fr{H_assump}, each of the vectors $\bH \bF_{k_1, k_2}$ has only one non-zero component, 
so that  the set $J$ has at most $\ga(1-\ga)m^2 < s$ nonzero elements.



Consider a collection of binary vectors $\bom \in \{0,1\}^{(1-\ga)m}$.
By Varshamov-Gilbert Lemma (see Tsybakov (2009)), there exists a subset $\Xi$ of those vectors 
such that for any $\bom, \bom' \in \Xi$ one has 
$\|\bom - \bom'\|_H = \|\bom - \bom'\|^2 \geq (1 - \ga)m/8 \geq m/16$
and $|\Xi| \geq \exp((1-\ga)m/8)$. 
Assume, without loss of generality, that $m$ is large enough, so that 
$\exp((1-\ga)m/8) \geq \ga m$, otherwise, choose a smaller value of $\ga$ 
(inequality $\exp((1-\ga)m/8) \geq \ga m$ is always valid for $\ga \leq 1/9$).
Choose $\ga m$ vectors $\bom$ in $\Xi$, enumerate them as $\bom^{(1)}, \cdots, \bom^{(\ga m)}$
and use them to form columns of matrix $\bF$ as follows:
\be \label{Fmatr}
\bF_{*,j} = 0.5\, \bone + \al \bom^{(j)}, \quad j=1, \cdots, \ga m.
\ee
Then, for any $j, j' = 1, \cdots, \ga m$, obtain
\be \label{dist1}
\|\bF_{*,j} - \bF_{*,j'}\|^2 \geq \al^2   m/16,
\ee
where $\al$ is a positive constant that will be defined later.
%
%
Note that for every $l$ and $k$ one has 
\be \label{inequalities}
l\, \log\lkr \frac{k}{l}\rkr \leq \log {k \choose l} \leq l\, \log\lkr \frac{k e}{l}\rkr, \quad 
\log(k!) = k \log k - k + \frac{1}{2} \log(2\pi k) + o(1),
\ee
where the $o(1)$ term is smaller than 1. Therefore, it follows from the first formula in \fr{card_clust} that
\be \label{log_card}
n \log m + n_0(L-1)\, \log \lkr \frac{m n}{n_0} \rkr \leq 
\log|\calZ (m,n,n_0,L)| \leq  
n \log m + n_0(L-1)\, \log \lkr \frac{m n e}{n_0} \rkr  
\ee
The term $n \log m$ in \fr{log_card} is due to the initial clustering while the term 
$n_0(L-1)\, \log \lkr  m n/n_0  \rkr$ is due to temporal changes in the clusters' memberships.


In what follows, we shall utilize clustering functions $z^{(l)}: [n] \to [m]$
corresponding to clustering matrices $\bC^{(l)}$  such that  $z^{(l)} (j) = k$ 
iff at the moment $t_l$ node $j$ belongs to class $\Om_k$, $k=1, \cdots, m$. 
% Denote a set of such clustering matrices at time $t_l$ by $\calZ_l (m,n,n_0)$ and the 
% overall collection   by $\calZ (m,n,n_0,L)$.   Note that $\calZ_1 (m,n,n_0)$ does not depend on   $n_0$,
% so that   $\calZ_1 (m,n,n_0)  \equiv  \calZ_1 (m,n)$.
 \\

 

%%%%%%%%%%%%%%%%%%%%%%%%%%%%%%%%%%%%%%%%%%%%%%%%%%%%%%%%%%%%%%%%%%%%%%%%%%%%%%%%%%%%%%%%%%%%%%%

{\bf Clustering error due to initial clustering. } 
First consider the case when initial clustering error dominates. 
If $m=2$ or $m$ takes a small value, the proof is almost identical to the proof in Section 3.3 of  Gao \etal (2015).
Hence, we shall skip this part and consider the case when $m$ is large enough,
so that $\ga m \geq 2$.
 
Following  Gao \etal  (2015), we consider   clustering matrices and clustering functions 
independent of $l$, so that $z^{(l)}\equiv z$.
% \bes
% z^{(l)}\equiv z, \quad \calZ (m,n) = \calZ_1 (m,n)  
% \ees 
Consider a sub-collection of clustering matrices $\calF(m,n,\ga) \subset \calM(m,n)$
such that they cluster the first $n(1-\ga)$ nodes into the first $m(1-\ga)$ classes uniformly and sequentially,
$n/m$ nodes in each class, i.e., the first $n/m$ nodes are placed into class  $\Om_1$, 
the second $n/m$ nodes into class  $\Om_2$, and so on.
%
The remaining $\ga n$ nodes are clustered into the remaining $\ga m$ classes, $n/m$  nodes into each class.
Then, by Lemma \ref{lem:cardinality}, 
$$
\log |\calF(m,n,\ga)| = \log \lkr (\ga n)!\Big/ \lkv (n/m)!\rkv^{\ga m} \rkr \geq  \ga n  \log(\ga m)/4.
$$  
Now, apply Lemma~\ref{lem:packing} with $\ga n$ and $\ga m$, respectively, instead of $n$ and $m$ and 
$r = \ga n/32$. Derive  that there exists a 
subset $\calS(m,n,\ga)$ of the set $\calF(m,n,\ga)$ such that, for any $\bC, \bC' \in \calS(m,n,\ga)$, one has 
$2 \lfi \#j:\  z(j) \neq z'(j)\rfi = \| \bC  - \bC' \|_H   \geq \ga n/32$. Also,  by \fr{AA3}, 
\be    \label{cardSnm}
\log |\calS(m,n,\ga)| \geq \frac{\ga n}{4} \log (\ga m)  -   \frac{\ga n\,  \log (32 m \ga  e)}{32}  \geq
 \frac{\ga n}{16} \log (\ga m).
\ee 
%
% 
Let $\bLam$ and $\bLam'$ be the tensors of probabilities 
corresponding to,  respectively, clustering matrices 
$\bC, \bC' \in \calS(m,n,\ga)$ with related clustering functions
$z$ and $z'$. Then, by \fr{dist1},
due to the fact that the first $n(1-\ga)$ nodes are clustered uniformly and 
sequentially,   obtain
\begin{align*}
& \| \bLam - \bLam'\|^2 
  =   2 L \, \sum_{i=1}^{n(1-\ga)}\, \sum_{j = n(1-\ga)+1}^n (\bF_{z(i),z(j)} -  \bF_{z'(i),z'(j)})^2  \\
%
&  =    \frac{2 L n}{m} \, \sum_{k=1}^{m(1-\ga)}\, \sum_{j = n(1-\ga) + 1}^n (\bF_{k,z(j)} -  \bF_{k,z'(j)})^2 \\
%
& =     \frac{2 L n}{m} \,   \sum_{j = n(1-\ga) + 1}^n \|\bF_{*,z(j)} -  \bF_{*,z'(j)}\|^2  
%
  \geq  \frac{2 L n}{m} \,   \frac{\al^2\, m}{16} \ \lfi \#j:\  z(j) \neq z'(j)\rfi,   
\end{align*}
so that 
\be \label{d1}
\| \bLam - \bLam'\|^2  \geq 2^{-9}\, L n^2 \al^2 \ga.
\ee
On the other hand, if $\al \leq 1/4$, then, by Lemma~\ref{lem:Gao2015},
obtain that the Kullback divergence is bounded above 
\be \label{kul1}
K (\PP_{\bLam}, \PP_{\bLam'}) \leq 8\, \| \bLam - \bLam'\|^2  \leq
16 \al^2 n^2 L  \ga.
\ee 
Set $\al^2 = C_\al \, \log (\ga m)/nL$ % where $C_\al$ is an absolute constant
 and apply Theorem 2.5 of Tsybakov (2009).
Due to \fr{cardSnm} and \fr{kul1}, if $C_\al$ is a small enough absolute constant,  
$$
16 \al^2 n^2 L \ga = 16 C_\al \, \log (\ga m) n < (1/8)\, (\ga n/16)\, \log (\ga m) \leq (1/8)\,  \log |\calS(m,n,\ga)| 
$$
and conditions of   Theorem 2.5 are satisfied. 
Since  
$L n^2 \al^2 \ga = C_\al  n\,\ga \log (\ga m)$ and $\log (\ga m) \geq C(\ga) \log m$ for some constant $C(\ga)$ dependent on $\ga$ only, 
derive  
\be \label{lower_DSBM1}
\inf_{\hbLam} \sup_{\stackrel{\bG \in \calG_{m,L,s}}{\bC \in \calS(m,n,\ga)}}
\PP_{\bLam} \lfi \frac{\|\hbLam - \bLams\|^2}{n^2\,L} \geq   \frac{C(\ga)\, \log m}{n L}
\rfi \geq 1/4.
\ee
 

%%%%%%%%%%%%%%%%%%%%%%%%%%%%%%%%%%%%%%%%%%%%%%%%%%%%%%%%%%%%%%%%%%%%%%%%%%%%%%%%%%%%%%%%%%%%%%%%%%%%%%%%%%%%%%%%%%%%

 
 

{\bf Clustering error due to  changes in the memberships. }
Now, we consider the case when the clustering error which is due to the temporal changes in memberships dominates
the error of initial clustering. Use the same construction for $\bG$ and $\bF$ as before. 
Consider the following collection of clustering matrices $\calF = \prod_{l=1}^L \calF_l$ where $\calF_l$ is defined as follows.
When $l$ is odd, $\calF_l$ contains only one matrix that clusters nodes uniformly and sequentially, i.e., 
the first $n/m$ nodes go to class $\Om_1$, the second  $n/m$ nodes go to class $\Om_2$
and the last $n/m$ nodes go to class $\Om_m$. If $l$ is even, $\calF_l =  \calP(m,n,n_0,\ga)$ where $\calP(m,n,n_0,\ga)$ 
is the set of clustering matrices  that corresponds to a perturbation of the uniform sequential 
clustering with at most $n_0$ nodes moved to different classes 
in the manner described below. Let $k_0$ be an integer such that
\be \label{k0}
k_0 \leq n_0/(\ga m) < k_0 +1.
\ee
If $k_0 =0$, then $n_0 < \ga m$ and we choose $n_0$ clusters out of the last $\ga m$ clusters,
remove one element from each of those clusters and then put those $n_0$ elements back  
in such a manner that every element goes to a different cluster and 
no elements goes back to its own cluster. If $k_0 \geq 1$, we remove $k_0$ elements from each of the 
last $\ga m$ clusters and then put each of those $k_0$-tuples back, one tuple per cluster, 
so that none of the tuple goes back to its own cluster.
Then, $\log |\calF| = [L/2] \log|\calP(m,n,n_0, \ga)|$ where $[L/2] \geq (L-1)/2$ is the largest integer not exceeding $L/2$ and 
\bes % \label{F_card}
\log|\calP(m,n,n_0, \ga)| = \lfi
\begin{array}{ll}
\log {\ga m \choose n_0} + n_0 \log(n/m) + \log[(n_0-1)!], & \mbox{if}\quad k_0 =0;\\
\ga m \log {n/m \choose k_0} + \log[(\ga m -1)!], & \mbox{if}\quad k_0 \geq 1.
\end{array} \right.
\ees
If $n_0 < \ga m$, so that $k_0 =0$,  then, by \fr{inequalities}, obtain that
$\log|\calP(m,n,n_0, \ga)| \geq n_0 \log(\ga m/n_0) + n_0 \log(n/m) = n_0 \log(\ga n/n_0)$.
If $n_0 \geq \ga m$ and $k_0 \geq 1$, then,  
by \fr{inequalities}, obtain
$\log|\calP(m,n,n_0, \ga)| \geq \ga   m k_0\,  \log(n/(m k_0))$.
Since $k_0 +1 \leq 2 k_0$, obtain that $k_0 \geq n_0/(2 m \ga)$. Hence, for any $k_0 \geq 0$ 
\be \label{calF_card}
\log |\calP(m,n,n_0, \ga)| \geq \frac{n_0}{2} \, \log \lkr \ga n/n_0  \rkr.
\ee
%
%
For every even value of $l$, apply Lemma \ref{lem:packing} with $\ga n$ and $\ga m$, 
respectively, instead of $n$ and $m$ and $r = n_0/40$ obtaining that there exists a 
subset $\calS_l (m,n,n_0,\ga)$ of the set $\calP(m,n,n_0, \ga)$ such that, for any 
$\bC^{(l)}, \bC'^{(l)} \in  \calS_l (m,n,n_0,\ga)$, one has 
\be \label{d_l}
  \| \bC^{(l)}  - \bC'^{(l)} \|_H   \geq  n_0/40, \quad  l=2k,\ k=1, \cdots, [L/2].
\ee 
By \fr{calF_card} and Lemma~\ref{lem:packing}, for every even $l$, one has 
\begin{align*}
\log |\calS_l (m,n,n_0,\ga)| \geq    \frac{n_0}{2}\, \log\lkr \frac{\ga n}{n_0}\rkr -   
\frac{n_0}{40}\, \log \lkr \frac{80 n e m \ga^2}{n_0}\rkr \geq \frac{n_0}{40}\,  \log \lkr \frac{n e m}{n_0}\rkr
\end{align*}
since, due to $n_0 \leq   4/3\, \ga n\,m^{-1/9}$ and $(80 e^2)^{1/18} \leq 0.75$, one has 
$(\ga n/n_0)^{20} \ge (80 e^2)\, [m\, (\ga n /n_0)]^2$, so that 
$$
20\, \log(\ga n/n_0) \ge  \log(80 n^2 m^2 e^2 \ga^2/n_0^2) = \log(n e m/n_0) +
 \log (80 n e m \ga^2/n_0). 
$$   
% if $\ga n/n_0  \geq m^{1/9} (80 e^2)^{1/18} \leq 0.75\, m^{1/10}$. 
For odd values of $l$, let $\calS_l (m,n,n_0,\ga)$ contain just one clustering matrix corresponding to the 
uniform sequential clustering. 
%
%
Now, consider the set $\calS  (m,n,n_0,\ga,L) = \prod_{l=1}^{L} \calS_l (m,n,n_0,\ga)$ with   
\be  \label{cardSnmL}
% \calS  (m,n,n_0,\ga,L) = \prod_{l=1}^{L} \calS_l (m,n,n_0,\ga) \ \mbox{with} \ 
\log|\calS  (m,n,n_0,\ga,L)| \geq  \frac{(L-1)n_0}{80}\, \log \lkr \frac{n e m}{n_0}\rkr.
\ee 
Let $\bC = (\bC^{(1)}, \bC^{(2)},\cdots,\bC^{(L)})$ and $\bC' = (\bC'^{(1)}, \bC'^{(2)},\cdots,\bC'^{(L)})$  
be two sets of clustering matrices with $\bC^{(l)}, \bC'^{(l)} \in  \calS_l (m,n,n_0,\ga)$ and let $z = (z_1, \cdots, z_L)$ and 
 $z' = (z'_1, \cdots, z'_L)$  be the corresponding clustering functions.
Let $\bLam$ and $\bLam'$ be the tensors of probabilities 
corresponding to sets of clustering matrices $\bC, \bC' \in \calS  (m,n,n_0,\ga,L)$. 
% and related clustering functions $z$ and $z'$. 
%
Then, similarly to the previous case, using \fr{dist1}, derive
\begin{align*}
& \| \bLam - \bLam'\|^2 
  =   2 \sum_{l=1}^{[L/2]} \, \sum_{i=1}^{n(1-\ga)}\, \sum_{j = n(1-\ga)+1}^n (\bF_{z_{2l}(i),z_{2l}(j)} -  \bF_{z'_{2l}(i),z'_{2l}(j)})^2  \\
%
& =    \frac{2n}{m} \, \sum_{l=1}^{[L/2]} \, \sum_{k=1}^{m(1-\ga)}\, \sum_{j = n(1-\ga)+1}^n (\bF_{k,z_{2l}(j)} -  \bF_{k,z'_{2l}(j)})^2 \\
%
& =       \frac{2n}{m} \, \sum_{l=1}^{[L/2]} \,  \sum_{j = n(1-\ga)+1}^n \|\bF_{*,z_{2l}(j)} -  \bF_{*,z'_{2l} (j)}\|^2  
%
  \geq    \frac{n \al^2}{8} \, \sum_{l=1}^{[L/2]}   \|\bC^{(2l)} - \bC'^{(2l)}\|_H, 
\end{align*}
so that  by \fr{d_l}, 
\be \label{d11}
\| \bLam - \bLam'\|^2  \geq  (L-1) n n_0 \al^2 /1280.
\ee
%
%
Again, similarly to the previous case, if $\al \leq 1/4$, then by Lemma~\ref{lem:Gao2015},
obtain that the Kullback divergence is bounded above 
\be \label{kul2}
K (\PP_{\bLam}, \PP_{\bLam'}) \leq 8\, \| \bLam - \bLam'\|^2  \leq
8  L n n_0 \al^2.
\ee 
Set 
%\bes
$\al^2 = C_\al n^{-1}\, \log (n e m/n_0)$
%\ees
 where $C_\al$ is an absolute constant  and apply Theorem 2.5 of Tsybakov (2009).
%
Observe that if $C_\al$ is small enough, then, due   to \fr{cardSnmL} and \fr{kul2},  conditions of this theorem are satisfied, hence,  
\be \label{lower_DSBM2}
\inf_{\hbLam} \sup_{\stackrel{\bG \in \calG_{m,L,s}}{\bC \in \calS  (m,n,n_0,\ga,L)}}
\PP_{\bLam} \lfi \frac{\|\hbLam - \bLams\|^2}{n^2\,L} \geq  C(\ga) \,   \frac{n_0}{n^2}\,\log \lkr \frac{n e m}{n_0} \rkr
 \rfi \geq 1/4.
\ee
 
 

%%%%%%%%%%%%%%%%%%%%%%%%%%%%%%%%%%%%%%%%%%%%%%%%%%%%%%%%%%%%%%%%%%%%%%%%%%%%%%%%%%%%%%%%%%%%%%%%%%%%%%%%%%%%%%%%%%%%

 
\underline{\bf The nonparametric estimation error. } 
Consider uniform sequential clustering with $n/m$ nodes in each group and 
group memberships remaining the same for all $l=1, \cdots, L$. 
% Let tensor $\bG$ be such that $\bG_{k,k,l} =0$ for any $k = 1, \cdots, m$, $l=1, \cdots, L$. 
Let $\bQ  \in \RR^{M\times L}$ be the matrix with columns   $\bq^{(l)}$,
$l=1, \cdots, L$, defined in Section \ref{sec:vectorization}.
Denote $\bV = \bQ \bH^T \in \RR^{M\times L}$ and recall that for $\bG \in \calG_{m,L,s}$,   by   \fr{vecQHT}
and \fr{transformed}, matrix $\bV$  should have at most $s$ nonzero entries.  


Let $k_0 = \min(s/2, M)$.  Choose $k_0$ rows among $M$ rows of matrix $\bV$
and denote this set by $\calX$. If $k_0 = M$,  set  $\calX  = \{1, \cdots, M\}$.
For $k \in \calX$, set $\bV_{k,1} \neq 0$. 
We have already distributed $k_0$ non-zero entries and have $s -k_0$ entries left.
We distribute those entries into  the $k_0$ rows $\bV_{k,*}$ where $k \in \calX$.
Let   
\be \label{k0s0} 
s_0 = [(s -k_0)/k_0] = [s/k_0] -1 \quad \mbox{with} \quad s/2 \leq k_0 s_0 <s,
\ee 
where $[s/k_0]$ is the largest integer no larger than $s/k_0$.
Consider a set of binary vectors $\bom \in \{0,1\}^{L}$ with exactly $s_0$ ones in each 
vector. By Lemma 4.10 of Massart (2007), there exists a subset $\calT$ of those vectors  such that 
for any $\bom, \bom' \in \calT$, one has
\bes
\|\bom - \bom'\|_H \geq s_0/2  \quad \mbox{and} \quad 
\log|\calT| \geq 0.233\, s_0\, \log(L/s_0).
\ees 
Denote $\tilde{\calT} = \bigotimes_{k \in \calX} \calT_k$, where $\calT_k$ is a copy of the set $\calT$
corresponding to row $k$ of matrix $\bV$.
%
For $\bom^{(k)} \in \calT_k$, set 
\be \label{bV}
\bV_{k,*} = (\sqrt{L}/2, \cdots, 0) + \al m/n \,  \bom^{(k)}, \ \mbox{if}\ k \in \calX,
\quad  \bV_{k,*} = 0 \ \mbox{if}\  k \notin \calX.
\ee
It is easy to see that matrix $\bV$ has at most $s$ nonzero entries as required. 



Let $\bV$ and $\bV'$ be matrices corresponding to sequences $\bom^{(k)}$ and $\bom'^{(k)}$  
in $\calT_k$, $k \in \calX$. Let $\bLam$ and $\bLam'$ be the tensors corresponding to $\bV$ and $\bV'$.
% It is pretty straightforward to check that $s/2 \leq k_0 s_0 \leq s$.   
Then, due to \fr{k0s0} and the uniform sequential clustering,   \fr{bV} implies that 
\begin{align*} 
\|\bLam - \bLam'\|^2 
& \geq   \lkr \frac{n}{m}\rkr^2 \|\bV - \bV'\|^2 
\geq   \lkr \frac{n}{m}\rkr^2 \sum_{k \in \calX} \|\bV_{k,*} - \bV'_{k,*}\|^2  
  \geq   \frac{\al^2  k_0 s_0}{2} \geq  \frac{\al^2  s}{4};\\
\|\bLam - \bLam'\|^2 & \leq    4\, \lkr \frac{m}{n}\rkr^2 \al^2 \lkr \frac{n}{m}\rkr^2 k_0 s_0 \leq 4 \al^2 s.
\end{align*}
Set $\al^2 = C_\al \log (L/s_0)$.    It is easy to check that, due  to assumptions \fr{H_assump} and \fr{Js_cond},
one has $\bQ_{ij} \in [1/4,3/4]$ for any $i$ and $j$. Hence,   by Lemma~\ref{lem:Gao2015}, obtain
$K (\PP_{\bLam}, \PP_{\bLam'}) \leq 8\, \| \bLam - \bLam'\|^2  \leq 32 \al^2 s$. 
 If $C_\al \leq 2^{-8} \cdot 0.233$, then $32 \al^2 s < (1/8) \log(\tilde{\calT})$ and 
conditions of Theorem 2.5 of Tsybakov (2009) hold.  


Finally, in order to obtain the last term in \fr{lower_DSBM}, examine $L/s_0$.
If $s < 2M$, then $k_0 = s/2$, $s_0=1$ and  $L/s_0 = L = L m^2/m^2 \geq \ga L m^2/s$.
If $s \geq 2M$, then $k_0 = M$, $s_0\leq s/M$ and $L/s_0 \geq LM/s \geq Lm^2/(2s) \geq \ga L m^2/s$. 
Since for some constant $C(\ga)>0$ independent of $L$ and $m$, one has $\log (\ga L m^2/s) \geq C(\ga) \log(L m^2/s)$, obtain
\be \label{lower_DSBM3}
\inf_{\hbLam} \sup_{\stackrel{\bG \in \calG_{m,L,s}}{\bC \in \calS  (m,n,n_0,\ga,L)}}
\PP_{\bLam} \lfi \frac{\|\hbLam - \bLams\|^2}{n^2\,L} \geq   C_\ga \,   \frac{s}{n^2 L}\,\log \lkr \frac{L  m^2}{s} \rkr
 \rfi \geq 1/4.
\ee
 


Finally, in order to obtain the lower bound in \fr{lower_DSBM} observe that, for any $a,b,c \geq 0$, 
one has $\max(a,b,c)\leq a+b+c \leq 3 \max(a,b,c)$     and then combine
\fr{lower_DSBM1}, \fr{lower_DSBM2} and \fr{lower_DSBM3}.
\\ 
 
 


%%%%%%%%%%%%%%%%%%%%%%%%%%%%%%%%%%%%%%%%%%%%%%%%%%%%%%%%%%%%%%%%%%%%%%%%%%%%%%%%%%%%%%%%%%%%%%%%%%%%%%%%%%%%%%%%%%%%
%%%%%%%%%%%%%%%%%%%%%%%%%%%%%%%%%%%%%%%%%%%%%%%%%%%%%%%%%%%%%%%%%%%%%%%%%%%%%%%%%%%%%%%%%%%%%%%%%%%%%%%%%%%%%%%%%%%%

  


\noindent
{\bf Proof of Theorem \ref{th:smooth_DSBM_lower_bounds}. }
Note that for Theorem~\ref{th:DSBM_lower_bounds} we proved the lower bounds for the clustering error in the most restrictive case when 
$\bC \in  \calZ_{\bal} (m,n,n_0,L,1,1)$. Moreover, in this proof, the connection probabilities are set to be constant over time,  hence
due to condition  \fr{H_assump}, $\bDs_{k,l}=0$ for $l \geq 2$ in assumption \fr{bias_coef_cond}, so \fr{bias_coef_cond} holds
for any $K_0$ and $\nu_0$. Therefore, the lower bounds for the risk due to clustering errors hold in this case and coincide with the lower bounds in 
Theorem~\ref{th:DSBM_lower_bounds}. For this reason, we only need to prove the lower bounds that are due to the nonparametric estimation error.
\\


%%%%%%%%%%%%%%%%%%%%%%%%%%%%%%%%%%%%%%%%%%%%%%%%%%%%%%%%%%%%%%%%%%%%%%%%%%%%%%%%%%%%%%%%%%%%%%%%%%%%%%%%%%%%%%%%%%%%

 
\underline{\bf The nonparametric estimation error. } 
Consider a set up where the nodes are grouped into $m$ classes,
$n/m$ nodes in each class, so the model is fully balanced.
Let $\bG_{k,k,l} =0$ for any $k=1, \cdots, m$ and $l=1, \cdots, L$.



Consider an even number $L_0$ such that $1 \leq L_0 \leq L/2$ and a set of vectors $\bom \in \{ 0,1\}^{L_0}$ 
with exactly $L_1 = L_0/2$ nonzero entries.
%
By Lemma 4.10 of Massart (2007), there exists a subset $\calT$ of those vectors such that 
for any $\bom, \bom' \in \calT$ one has 
\be \label{new_bombom}
\|\bom - \bom'\|_H \geq L_0/4, \quad
\log |\calT| \geq 0.233 (L_0/2) \log 2 \geq 0.08 L_0.
\ee 
Denote $\calK = \{(k_1, k_2):\ 1 \leq k_1 < k_2 \leq m\}$ and 
let  $\calT_{k_1, k_2}$ be the copies of $\calT$ for $(k_1, k_2) \in \calK$. 
Denote $\tilde{\calT} = \bigotimes_{(k_1,k_2) \in \calK}\, \calT_{k_1, k_2}$ and
observe that 
\bes
\log|\tilde{\calT}| = 0.5\, m(m-1)\, \log |\calT| \geq 0.02 m^2\, L_0.
\ees
Then, $\tilde{\bom} \in \tilde{\calT}$ are   binary tensors with elements $\bom_{l}^{(k_1, k_2)}$, $l=1, \cdots, L_0$,
and  $(k_1,k_2) \in \calK$.
Consider a set of matrices  $\bD^{(\tilde{\bom})}$ indexed by  $\tilde{\bom}$
such that for the index $k = 1, \cdots, M,$ corresponding to $(k_1,k_2) \in \calK$, one has
\bes
\bD^{(\tilde{\bom})}_{k,1} = \sqrt{L}/2; \quad 
\bD^{(\tilde{\bom})}_{k, l} =  
\al \, \bom_{l-L_0}^{(k_1, k_2)},\ l=L_0+1, \cdots, 2L_0  , \  k =(k_1,k_2) \in \calK.
% \quad \mbox{if}
% \beta_{k_1-1}  < x \leq \beta_{k_1}, \ \beta_{k_2-1} < y \leq \beta_{k_2}.
\ees  
% where $\tilde{\bom}$ is a binary matrix with elements $\bom_{l}^{(k_1, k_2)}$,$l=1, \cdots, L_0$
% and  $(k_1,k_2) \in \calK$.
%
In order condition \fr{bias_coef_cond} is satisfied, we set
\be \label{new_rhoL0}
\al^2 \leq C_1 L_0^{-(2 \nu_0 +1)}  \quad \mbox{with} \quad C_1 \leq  \min(K_2 2^{1 - 2\nu_0},1/8).
\ee 
Denote by $\bLam$ and $\bLam'$ the probability tensors corresponding, respectively,  to $\tilde{\bom}$ and 
$\tilde{\bom'}$ in $\tilde{\calT}$. Then, due to \fr{new_bombom} and the symmetry, 
\beqns
\|\bLam - \bLam'\|^2 &\geq &    \al^2 \sum_{k_1 =1}^m \sum_{k_2 = k_1 +1}^m 
\|\bom^{(k_1, k_2)} - \bom'^{(k_1, k_2)}\|_H \lkr \frac{n}{m}\rkr^2  \geq \frac{\al^2 n^2 L_0}{8};\\
\|\bLam - \bLam'\|^2 &\leq &    \al^2 m(m-1) (n/m)^2 L_0 \leq \al^2 n^2 L_0.
\eeqns
Note that  one has $1/4 \leq \bQ_{k,l} \leq 3/4$ provided
$\|\bH^T \bom ^{(k_1, k_2)}\|_\infty \leq 1/4$ for $(k_1, k_2) \in \calK$.
By  Assumption \fr{H_assump}, the latter is guaranteed by 
$\al^2 L_0^2/L \leq 1/4$, so that, due to $L  \geq 2 L_0$ and $\nu_0 \geq 1/2$,
it is ensured by \fr{new_rhoL0}. Then, by  Lemma~\ref{lem:Gao2015}, one has 
\bes
K (\PP_{\bLam}, \PP_{\bLam'}) \leq 8\, \| \bLam - \bLam'\|^2  \leq
8  \al^2 n^2 L_0 \leq \log |\tilde{\calT}|/8   
\ees  
provided 
\bes % \label{rho_con1}
\al^2 \leq C_2 (m/n)^2,
\ees
 where $C_2$ is an absolute constant.
Therefore, application of Theorem 2.5 of Tsybakov (2009) yields that 
\bes
\inf_{\hbLam} \sup_{\stackrel{\bG \in \calG_{m,L,s}}{\bC \in \calZ_{\bal}}}
\PP_{\bLam}   \lfi \frac{\|\hbLam - \bLam \|^2}{n^2\,L}   \geq \Del(n,L) \rfi \geq \frac{1}{4} 
\ees
with $\Del(n,m,L) = C\,\al^2 L_0/L$ where $C$ is an absolute constant.



Now, we denote $C_3^2 = (C_2/C_1) 2^{-(2 \nu_0 +1)}$ and consider two cases.   
If $n \leq C_3 m L^{\nu_0 + 1/2}$, choose $L_0 = [(C_1/C_2) (n/m)^2]^{1/(2 \nu_0 +1)}$ 
which leads to $\al^2 = C_1 L_0^{-(2 \nu_0 +1)} = C_2 (m/n)^2$.  
It is easy to check that $L_0 \geq 2$ and that $L_0 \leq L/2$, so that  
\bes 
\Del(n,m,L) =   \frac{C}{L} \lkv \lkr \frac{m}{n}\rkr^2   \rkv^{\frac{2 \nu_0}{2 \nu_0+1}}.
\ees
If  $n > C_3 m L^{\nu_0 + 1/2}$, choose $\al^2 = C_2 (m/n)^2$ and set $L_0 = L/2$. Then, 
\fr{new_rhoL0} holds and $\Del(n,m,L) = C (m/n)^2$ which completes the proof of the lower bound
in this case. 
\\
 
%%%%%%%%%%%%%%%%%%%%%%%%%%%%%%%%%%%%%%%%%%%%%%%%%%%%%%%%%%%%%%%%%%%%%%%%%%%%%%%%%%%%%%%%%%%%%%%%%5




 
  



%%%%%%%%%%%%%%%%%%%%%%%%%%%%%%%%%%%%%%%%%%%%%%%%%%%%%%%%%%%%%%%%%%%%%%%%%%%%%%%%%%%%%%%%%%%%%%%%%


\subsection{Proof of Theorem \ref{th:sparse} } 

% Let $\bC$ be a clustering matrix and $\PCJ$ be the projection matrix on the column space of matrix $(\bC \bW^T)_J$.
Denote $\bW^T = \bV$ and recall that $\bS^2 = \bC^T \bC$ where $\bS^2$ is the diagonal matrix with entries $N_k^{(l)}$, 
the number of nodes in the pair of classes $k = (k_1, k_2)$ at time $t_l$. 
Note that  $(\bC \bW^T)_J = \bC \bV_J$, so that $\PCJ = \bC \bV_J [\bV_J^T \bC^T \bC \bV_j]^{-1} \bV_J^T \bC^T$
can be written as 
\bes 
\PCJ = \bC \bS^{-1} \PSJ (\bC \bS^{-1})^T \quad \mbox{with} \quad \PSJ = \bS \bV_J (\bV_J^T \bS^2 \bV_J)^{-1} \bV_J^T \bS.
\ees
Here, $\PSJ$ is the projection matrix on the column space of matrix $\bS \bV_J$. 

%
Since $(\hbd, \hbC, \hm, \hJ)$ are solutions of optimization problem \fr{opt_problem_balanced},
for any   $J$  and $\bds = \vect(\bDs)$,  one has
\begin{align}  \label{new_main_ineq}
\|\ba - \hbC \hbW^T\, \hbd_{(\hJ)}\|^2 & + 2  \rho_n\,  \Pen(|\hJ|,\hm) \\
& \leq \|\ba - \bCs \bWs^T \bds_{(J)} \|^2 + 2  \rho_n\, \Pen(|J|,\ms), \nonumber
\end{align} 
It follows from \fr{true_model} that for any $m,J,\bd$ and $\bC$
\begin{align*} 
& \|\ba -   \bC \bW^T  \bd_{(J)}\|^2  = \|(\ba - \bCs \bWs^T \bds) +  (\bCs \bWs^T \bds -  \bC \bW^T \bd_{(J)})\|^2 \\
& =   \|\bxi\|^2 + \|\bC \bW^T \bd_{(J)} - \bCs \bWs^T \bds\|^2 + 2 \bxi^T (\bCs \bWs^T \bds -  \bC \bW^T \bd_{(J)}).
\end{align*}
Hence, plugging the last identity into the inequality \fr{new_main_ineq}, derive that
% for any $m$, $J$, $\bC$  and $\bd$
\begin{align} \label{new_main_ineq1}
\|\hbC \hbW^T \hbd_{(\hJ)} - \bCs \bWs^T \bds\|^2 & \leq \|\bCs \bWs^T \bds_{(J)} - \bCs \bWs^T \bds\|^2 \\
& +    \Del + \rho_n\, \Pen(|J|,\ms) - \rho_n\, \Pen(|\hJ|,\hm), \nonumber 
\end{align}
%
where, $\Del   = 2 |\bxi^T (\hbC \hbW^T \hbd_{(\hJ)} - \bCs \bWs^T \bds_{(J)})|$. 
Note that, due to \fr{projection} and \fr{hbd_solution}, one has  $\hbC \hbW^T \hbd_{(\hJ)} = \hPChJ\, \ba$ 
with $\ba$ is given by \fr{true_model}, $\bCs \bWs^T \bds_{(J)} = \PCJs \btes$ and $\bCs \bWs^T \bds = \btes$.  
Therefore,
\begin{align}
  & \Del    \leq  \Del_1 + \Del_2 + \Del_3, \label{new_Del123}\\
  \Del_1   = 2 |\bxi^T \PCJso \btes|, \quad %\nonumber \\
  & \Del_2    = 2 |\bxi^T \hPChJo \btes|, \quad  
  \Del_3   = 2  \bxi^T \hPChJ\, \bxi.  \nonumber
\end{align}
% Now, we need to find upper bounds for each of the terms in \fr{new_Del123}.

 
%%%%%%%%%%%%%%%%%%%%%%%%%%%%%%%%%%%%%%%%%%%%%%
%
In order to obtain  an upper bound for $\Del_1$ and $\Del_2$, note that by Bernstein inequality, for any $\bC$, $m$, $\bte$, $J$ and
for any $x>0$ with probability at least $1 - 2 e^{-x}$, one has 
\bes
2 |\bxi^T \PCJo \bte| \leq 2 \sqrt{2 x \rho_n\, \|\PCJo \bte \|^2 } + 4/3\,  \|\PCJo \bte \|_{\infty}\, x.
\ees
Due to $2 ab \leq a^2 + b^2$, obtain that with probability at least $1 - 2 e^{-x}$
\be \label{eq:Bernstein}
2 |\bxi^T \PCJo \bte| \leq 0.5\, \|\PCJo \bte \|^2 + 4\, \lkr \rho_n\,   +    \|\PCJo \bte \|_{\infty}/3 \rkr x
\ee 
%
%%%%%%%%%%%%%%%%%%%%%%%%%%%%%%%%%%%%%%%%%%%%%%
%
Applying \fr{eq:Bernstein}  to $\Del_1$ with  $\bte = \btes$ and  $x=t>0$,   using \fr{uni_sparse_cond} and keeping in mind that 
$\|\bLams\|_\infty = \|\btes\|_\infty \leq \rhons \leq \rho_n$, obtain
\be \label{new_Del1er}
\PP \lfi \Del_1 -  0.5\, \| \bCs \bWs^T \bds - \bCs \bWs^T \bds_{(J)} \|^2  - 4\,  \rho_n (1 + B_0/3) t \leq  0  \rfi \geq 1 - 2 e^{-t}.
\ee
%
%%%%%%%%%%%%%%%%%%%%%%%%%%%%%%%%%
%
In order to obtain an upper bound for $\Del_2$, note that, similarly to the proof of Theorem~\ref{th:oracle}, one has  
\bes
\|\hbC \hbW^T \hbd_{(\hJ)} - \bCs \bWs^T \bds\|^2   
% + \|\hPhJ\, \bxi\|^2 + \| (\bI_{NL} - \hPhJ)  \bCs \bWs^T \bds\|^2 
\geq  \| (\bI_{NL} - \hPChJ)  \bCs \bWs^T \bds\|^2 = \|\hPChJo \bCs \bWs^T \bds\|^2.
\ees
Hence,  due to condition \fr{uni_sparse_cond}, since $\|\btes\|_\infty \leq \rhons\leq \rho_n$, obtain
\begin{align*} 
& \PP \lfi \Del_2 -  0.5\,  \|\hbC \hbW^T \hbd_{(\hJ)} - \bCs \bWs^T \bds\|^2 \leq
4 \lkr   \rho_n    + 1/3\,  \|\hPChJo \btes \|_{\infty} \rkr   x   \rfi \geq \\ 
& \PP \lfi \Del_2 -   0.5\,  \| \hPChJo  \bCs \bWs^T \bds\|^2 - 4\,  \rho_n (1 + B_0/3) x \leq 0 \rfi   
\end{align*}
Set  $x  = t + R(\hm,\hJ,L)$ where $R(m,J,L)$ is defined in \fr{RDel}.
Applying inequality \fr{eq:Bernstein} with $\bte = \btes$ % \bCs \bWs^T \bds$ 
together with the union bound  over $m=1, \cdots, n$, $\bC \in \calZ_{\bal} (m,n,n_0,L,\aleph_1,\aleph_2)$ 
and $J$ with $|J| = 1, \cdots, ML,$ we derive % that with probability at least $1 - 6 e^{-t}$, one has     
\begin{align}  \label{new_Del2er}
\PP \lfi
\Del_2 \right. & -  0.5\,  \|\hbC \hbW^T \hbd_{(\hJ)} - \bCs \bWs^T \bds\|^2  \\
& \left. - 4\,  \rho_n (1    + B_0/3)
(t + R(\hm,\hJ,L)) \leq 0  \rfi \geq 1 - 6 e^{-t}. \nonumber
\end{align} 
%
%%%%%%%%%%%%%%%%%%%%%%%%%%%%%%%%%
%  
For an upper bound for $\Del_3$,  write   
$$
\Del_3   = 2  \bxi^T \hbC \hbS^{-1} \hPShJ (\hbC \hbS^{-1})^T \bxi    = 2  \boeta^T   \hPShJ    \boeta.
$$
Here, for any fixed $\hbC = \bC$, due to Corollary~\ref{cor:subgaus}, vector $\boeta = (\bC \bS^{-1})^T \bxi$
has independent sub-Gaussian components such that,  for any index $i$ corresponding to a pair of nodes 
$k = (k_1, k_2)$ at time $t_l$,  one has 
$$
\| \boeta_{i} \|_{\psi_2}^2 \leq   1.5\, e^2  \, \max( \|\btes\|_{\infty}, [N^{(l)}_{k}]^{-1}).
$$
Observe that, due to \fr{eq:Nkl} and \fr{balanced},    
$$
N^{(l)}_{{k_1},{k_2}} \geq  0.5\, n^{(l)}_{k_1} n^{(l)}_{k_2} \geq 0.5\, \aleph_1^2 (n/m)^2.
$$
Therefore, 
$$
\| \boeta_{i} \|_{\psi_2}^2 \leq  \rho_{m,n} \quad \mbox{where} \quad  \rho_{m,n}= 
\frac{3e^2}{2} \, \max \lkr \rhons,  \frac{2}{\aleph_1^2}\, \frac{m^2}{n^2}\rkr.
$$   
Therefore, by Lemma 5.5 of Vershynin (2012), obtain that there exists an absolute constant $C_0$ such that 
for any vector  $\bt$  one has 
$$
\EE [\exp(\boeta^T \bt)] \leq \exp(C_0 \rho_{m,n} \|\bt\|^2). 
$$
Applying Lemma~\ref{lem:subgaus_quadr_err} with $\sig^2 = 2 C_0 \rho_{m,n}$, derive that for any fixed 
$\bC$ and $J$, one has
\bes  
\PP \lfi 2 \boeta  \PSJ \boeta   -  4 C_0 \rho_{m,n}\, \lkr 2 |J| +   3 t \rkr \leq 0 \rfi \geq  1- e^{-t}.
\ees
Again, taking a union bound over $m=1, \cdots, n$, $\bC \in \calZ_{\bal} (m,n,n_0,L,\aleph_1,\aleph_2)$ 
and $J$ with $|J| = 1, \cdots, ML,$ we derive that, due to $\rhons \leq \rho_n$, for some absolute constant $\tCo$  one has
\begin{align}  \label{new_Del3er}
\PP \lfi \Del_3 - \tCo \max \lkr \rho_n, \frac{\hm^2}{n^2} \rkr\,  [R(\hm,\hJ,L) + t] \leq 0 \rfi 
\geq  1- e^{-t}. 
\end{align}
where $R(m,J,L)$ is defined in \fr{RDel}.
The rest of the proof  of \fr{oracle_prob_sparse1} is very similar to the proof of Theorem~\ref{th:oracle}.
In order to establish \fr{oracle_prob_sparse2}, follow the arguments of the proof of Corollary~\ref{cor:upper_smooth_DSBM}.
 




 
%%%%%%%%%%%%%%%%%%%%%%%%%%%%%%%%%%%%%%%%%%%%%%%%%%%%%%%%%%%%%%%%%%%%%%%%%%%%%%%%%%%%%%%%%%%%%%%%%
%%%%%%%%%%%%%%%%%%%%%%%%%%%%%%%%%%%%%%%%%%%%%%%%%%%%%%%%%%%%%%%%%%%%%%%%%%%%%%%%%%%%%%%%%%%%%%%%%


\subsection{ Proof  of Theorem~\ref{th:upper_graphon}, Corollary~\ref{cor:upper_graphon} and 
Theorem~\ref{th:lower_graphon} }


This section contains proofs for the upper and the lower bounds for the risks in the case  of  graphon estimation.
\\

%%%%%%%%%%%%%%%%%%%%%%%%%%%%%%%%%%%%%%%%%%%%%%%%%%%%%%%%%%%%%%%%%%%%%%%%%
  
\noindent
{\bf Proof of Theorem \ref{th:upper_graphon}. }
To prove \fr{graph_upper}, we approximate the graphon by the DSBM and use inequality \fr{oracle_expec} in  Theorem \ref{th:oracle}.
In order to find an upper bound for the bias term, we need to cluster the nodes and create an approximate 
connectivity tensor $\bQ$. For this purpose, let $h \geq 1$ be a positive integer and denote
$\kappa_j = 1 + [(\beta_j - \beta_{j-1})h]$ where $[x]$ is the largest integer
no larger than $x$.  For $k = 1, \cdots, \kappa_j$ and $j = 1, \cdots, r$, consider a set of intervals 
\bes
U_{j,k}= (U_{j,k}^{(1)}, U_{j,k}^{(2)}] \quad \mbox{with} \quad  U_{j,k}^{(1)} =  \beta_{j-1} + (k-1)/h, \quad 
U_{j,k}^{(2)} = \min\{\beta_{j-1} + k/h, \beta_j\}.
% \quad k = 1, \cdots, \kappa_j; \ j = 1, \cdots, r.
\ees
Intervals $U_{j,k}$ subdivide every interval $(\beta_{j-1}, \beta_j]$ 
into $\kappa_j$ sub-intervals of  length at most $1/h$ and the total number of intervals 
is equal to 
\bes 
m = \sum_{j=1}^r \kappa_j.
\ees
Re-number the intervals consecutively as $U_1, \cdots, U_m$ and observe that since 
$(\beta_j - \beta_{j-1})h \leq \kappa_j \leq 1 +  (\beta_j - \beta_{j-1})h$,
one has 
\be \label{sr}
h \leq m \leq h+r.
\ee
The value $m$ in \fr{sr}  acts   as a number of classes.  Indeed, if $\zeta_i \in U_k$,
we place node $i$ into class $\Om_k$ and set $\bZ_{i, j} = \II(j=k)$. 


Let $\bTe^*$ be the true tensor of connection probabilities. Set $\bPhis = \bTes \bH^T$, 
$\bQ  = (\bZ^T \bZ)^{-1} \bZ^T \bTe^*$ and  $\bV  = \bQ  \bH^T$. 
%
Since $\bH$ is an orthogonal matrix,  the bias term  $\| \bC \bW^T \bd_{(J)} - \btes \|^2$ 
in the oracle inequality \fr{oracle_expec} is equal to 
\begin{align}  \nonumber
& \| \bC \bW^T \bd_{(J)} - \btes \|^2   =   \|\bZ \bV^{(\rho)} \bH - \bTes \|^2 = \|\bZ \bV^{(\rho)} - \bTe \bH^T\|^2\\
\label{bias_all} 
& =   \|\bZ \bV^{(\rho)} -  \bPhis \|^2 =     \|\bZ \bV^{(\rho)} -  \bPhis^{(\rho)}\|^2
+ \| \bPhis - (\bPhis)^{(\rho)}\|^2.
\end{align} 
The first term in the right-hand side of \fr{bias_all} describes how well the first $L^\rho$ columns of matrix $\bV^{(\rho)}$
represent the first $L^\rho$ columns $(\bPhis)^{(\rho)}$ of matrix $\bPhis$. The second term 
$\| \bPhis - (\bPhis)^{(\rho)}\|^2 = \| \bTes - (\bPhis)^{(\rho)} \bH \|^2$ evaluates 
how well $\bTes$ is represented by $L^\rho$ columns of its coefficients in the transform $\bH$.

The upper bound  for $\|\bZ \bV^{(\rho)} -  \bPhis^{(\rho)}\|^2$   can be found by repeating the calculations in Lemma 2.1 of Gao \etal (2015) 
with the only difference that $f$ is replaced by $v_l$ and there is an additional sum over $l = 1, \cdots L^\rho$.
Then, we obtain 
\be \label{bias1}
 \|\bZ \bV^{(\rho)} -  \bPhis^{(\rho)}\|^2  \leq K_1  2^{2 \nu_1} n^2 L^\rho\,  h^{-2 \nu_1}.
\ee
On the other hand, if $\rho <1$, then by Assumption {\bf A},
\be \label{bias2}
\| \bPhis - (\bPhis)^{(\rho)}\|^2 \leq \sum_{i_1=1}^n\, \sum_{i_2 =1}^n\ \sum_{l = L^\rho +1}^L \bv_l^2 (\zeta_{i_1}, \zeta_{i_2}) 
\leq K_2 n^2 L^{-2 \nu_2 \rho}
\ee
and $\| \bPhis - (\bPhis)^{(\rho)}\|^2 =0$ if $\rho=1$. 
Now, note that  for  given $m$ and $\rho$,  one has $|J| \leq m^2 L^\rho$, so that 
\be \label{gra_pen} 
\Pen(|J|,m) \leq C \lkv n \log m +   m^2 L^\rho \log(25\, L^{1-\rho})\rkv.
\ee  
Therefore,   \fr{sr}--\fr{gra_pen}  yield  \fr{graph_upper}.
\\
 

%%%%%%%%%%%%%%%%%%%%%%%%%%%%%%%%%%%%%%%%%%%%%%%%%%%%%%%%%%%%%%%%%%%%%%%%%%%%%%%%%%%%%%%%%%%%%%%%%%%%%%%%%%%%%%%%%%%%

%\medskip

\noindent
{\bf Proof of Corollary \ref{cor:upper_graphon}.  }
Note that if    $\nu_1 =\infty$, one can set $h=2$, so that the first term in \fr{graph_upper} is equal to zero, 
$h+r \leq 3 r$ and 
\bes  
\frac{\EE \|\hbLam - \bLams\|^2}{n^2\,L} \leq C \lfi \frac{I(\rho <1)}{L^{2 \rho \nu_2+1}} + 
\lkr\frac{r}{n}\rkr^2 \frac{1 + (1 - \rho)\log L}{L^{1-\rho}}  + \frac{\log r}{n L}\rfi.
\ees
Minimizing this expression with respect to $\rho \in [0,1]$ obtain the result in  \fr{graph_upper_cases} for $r = r_{n,L}$.
%
If $r=r_0$, then 
\be \label{risk2}
\frac{\EE \|\hbLam - \bLams\|^2}{n^2\,L} \leq C \lfi  
\frac{L^{\rho-1}}{h^{2 \nu_1}} + \frac{I(\rho <1)}{L^{2 \rho \nu_2+1}} + \frac{h^2\log L}{n^2 L^{1-\rho}}  
+ \frac{\log h}{n  L}  \rfi.
\ee 
Minimizing \fr{risk2} with respect to $h$ and $\rho$, obtain that the   values $h^*$ and $L^* = L^{\rho^*}$  delivering the minimum in 
\fr{risk2} are such that    $h^* \asymp  (n^2/\log L)^{1/(2(\nu_1+1)}$.
Hence, for some absolute constants $C$, one has  $\log (h^*) \leq C \log n$  and 
\bes
L^* =   L^{\rho^*} = \min \lfi L, C\, \lkr  n^2/\log L \rkr^{\frac{\nu_1}{(\nu_1 +1)(2 \nu_2+1)}}\rfi. 
\ees
Therefore,  \fr{graph_upper_cases} holds for $r = r_0$.
\\

 

 
%%%%%%%%%%%%%%%%%%%%%%%%%%%%%%%%%%%%%%%%%%%%%%%%%%%%%%%%%%%%%%%%%%%%%%%%%%%%%%%%%%%%%%%%%%%%%%%%%%%%%%%%%%%%%%%%%%%%
%%%%%%%%%%%%%%%%%%%%%%%%%%%%%%%%%%%%%%%%%%%%%%%%%%%%%%%%%%%%%%%%%%%%%%%%%%%%%%%%%%%%%%%%%%%%%%%%%%%%%%%%%%%%%%%%%%%%


% \subsection{Proof of Theorem \ref{th:lower_graphon}}
%\label{sec:lower}


  

 \noindent
 {\bf Proof of Theorem \ref{th:lower_graphon}. } 
We consider the cases when $r = r_{n,L}$ and $r=r_0$,
corresponding to piecewise constant and piecewise smooth graphon, separately.
\\
 

\underline{\bf Piecewise constant graphon. }
Assume, without loss of generality, that $j = n/r$ is an integer.  
Consider a set up where the nodes are grouped into $r$ classes and 
values of $\zeta_j$'s are fixed:   
\bes
\zeta_{k j + i} = \beta_k + (i-1/2)  (\beta_{k+1} - \beta_k)/j,
\quad k = 0, \cdots, r-1, \ i = 1, \cdots, j.
\ees
Then, there are $j$ nodes in each class.   
Let $\bG_{k,k,l} =0$ for any $k=1, \cdots, r$ and $l=1, \cdots, L$.



Consider an even number $L_0$ such that $1 \leq L_0 \leq L/2$ and a set of vectors $\bom \in \{ 0,1\}^{L_0}$ 
with exactly $L_1 = L_0/2$ nonzero entries.
%
By Lemma 4.10 of Massart (2007), there exists a subset $\calT$ of those vectors such that 
for any $\bom, \bom' \in \calT$ one has 
\be \label{bombom}
\|\bom - \bom'\|_H \geq L_0/4, \quad
\log |\calT| \geq 0.233 (L_0/2) \log 2 \geq 0.08 L_0.
\ee 
Denote $\calK = \{(k_1, k_2):\ 1 \leq k_1 < k_2 \leq r\}$ and 
let  $\calT_{k_1, k_2}$ be the copies of $\calT$ for $(k_1, k_2) \in \calK$. 
Denote $\tilde{\calT} = \bigotimes_{(k_1,k_2) \in \calK}\, \calT_{k_1, k_2}$ and
observe that 
\bes
\log|\tilde{\calT}| = r(r-1)/2 \log |\calT| \geq 0.02 r^2\, L_0.
\ees
Then, $\tilde{\bom} \in \tilde{\calT}$ are   binary tensors with elements $\bom_{l}^{(k_1, k_2)}$,$l=1, \cdots, L_0$
and  $(k_1,k_2) \in \calK$.
Consider a set of functions $f^{(\tilde{\bom})}$ indexed by  $\tilde{\bom}$
such that, for $\beta_{k_1-1}  < x \leq \beta_{k_1}$ and $\beta_{k_2-1} < y \leq \beta_{k_2}$,
their coefficients in the transform $\bH$ are given by 
\bes
\bv_1^{(\tilde{\bom})} (x,y) = \sqrt{L}/2; \quad \bv_l^{(\tilde{\bom})}  (x,y) =  
\al  \bom_{l-L_0}^{(k_1, k_2)},\ l=L_0+1, \cdots, 2L_0  , \  (k_1,k_2) \in \calK.
% \quad \mbox{if}
% \beta_{k_1-1}  < x \leq \beta_{k_1}, \ \beta_{k_2-1} < y \leq \beta_{k_2}.
\ees  
% where $\tilde{\bom}$ is a binary matrix with elements $\bom_{l}^{(k_1, k_2)}$,$l=1, \cdots, L_0$
% and  $(k_1,k_2) \in \calK$.
%
Then, Assumption \fr{A1} holds. In order condition \fr{A2} is satisfied, we set
\be \label{rhoL0}
\al^2 \leq C_1 L_0^{-(2 \nu_2 +1)}  \quad \mbox{with} \quad C_1 \leq  \min(K_2 2^{1 - 2\nu_2},1/8).
\ee 
Denote by $\bLam$ and $\bLam'$ the probability tensors corresponding, respectively,  to $\tilde{\bom}$ and 
$\tilde{\bom'}$ in $\tilde{\calT}$. Then, due to \fr{bombom} and the symmetry, 
\beqns
\|\bLam - \bLam'\|^2 &\geq &  2 \al^2 \sum_{k_1 =1}^r \sum_{k_2 = k_1 +1}^r 
\|\bom^{(k_1, k_2)} - \bom'^{(k_1, k_2)}\|_H \lkr \frac{n}{r}\rkr^2  \geq \frac{\al^2 n^2 L_0}{8};\\
\|\bLam - \bLam'\|^2 &\leq &    \al^2 r(r-1) (n/r)^2 L_0 \leq \al^2 n^2 L_0.
\eeqns
Note that  one has $1/4 \leq f(x,y,t) \leq 3/4$ provided
$\|\bH^T \bom ^{(k_1, k_2)}\|_\infty \leq 1/4$ for for $(k_1, k_2) \in \calK$.
By  Assumption \fr{H_assump}, the latter is guaranteed by 
$\al^2 L_0^2/L \leq 1/4$, so that, due to $L  \geq 2 L_0$ and $\nu_2 \geq 1/2$,
it is ensured by \fr{rhoL0}. Then, by Lemma~\ref{lem:Gao2015}, one has 
\bes
K (\PP_{\bLam}, \PP_{\bLam'}) \leq 8\, \| \bLam - \bLam'\|^2  \leq
8  \al^2 n^2 L_0 \leq \log |\tilde{\calT}|/8   
\ees  
provided 
\be \label{rho_con1}
\al^2 \leq C_2 (r/n)^2
\ee
 where $C_2$ is an absolute constant.
Therefore, application of Theorem 2.5 of Tsybakov (2009) yields \fr{graph_lower_ineq}
with 
$\Del(n,L) = C\,\al^2 L_0/L$ 
where $C$ is an absolute constant.



Now, we denote $C_3^2 = (C_2/C_1) 2^{-(2 \nu_2 +1)}$ and consider two cases.   
If $n \leq C_3 r L^{\nu_2 + 1/2}$, choose $L_0 = [(C_1/C_2) (n/r)^2]^{1/(2 \nu_2 +1)}$ 
which leads to $\al^2 = C_1 L_0^{-(2 \nu_2 +1)} = C_2 (r/n)^2$.  
It is easy to check that $L_0 \geq 2$ and that $L_0 \leq L/2$, so that  
\bes 
\Del(n,L) =   \frac{C}{L} \lkv \lkr \frac{r}{n}\rkr^2   \rkv^{\frac{2 \nu_2}{2 \nu_2+1}}.
\ees
If  $n > C_3 r L^{\nu_2 + 1/2}$, choose $\al^2 = C_2 (r/n)^2$ and set $L_0 = L/2$. Then, 
\fr{rhoL0} holds and $\Del(n,L) = C (r/n)^2$ which completes the proof of the lower bound 
when $r = r_{n,L}$.
\\
 
%%%%%%%%%%%%%%%%%%%%%%%%%%%%%%%%%%%%%%%%%%%%%%%%%%%%%%%%%%%%%%%%%%%%%%%%%%%%%%%%%%%%%%%%%%%%%%%%%5


\underline{\bf Piecewise smooth graphon. }
Since $r=r_0$ is a fixed quantity, without loss of generality, we set $r=1$.
Let  $\zeta_j = j/n, j=1, \cdots, n$, be fixed. Let $h$   be a positive integer, $1 \leq h \leq n$,
and denote $\del = 1/h$. Consider a kernel function $F(x)$ such that 
$F(x)$ is $n_1 > \nu_1$ times continuously differentiable  and for any $x,x' \in \RR$ and some $C_F >0$
\be \label{kern_properties}
\supp (F) = (-1/2; 1/2), \quad |F(x) - F(x')| \leq C_F |x  - x'|^{\nu_1}.
\ee
It follows from \fr{kern_properties} that $|F(x)| \leq C_F$ for any $x$.
Denote % $u_k = (k-1)\del + \del/2$, $k=1, \cdots, h$ and 
\be \label{Psi}
\Psi_{k_1, k_2} (x,y) = h^{-\nu_1} F(h(x - u_{k_1})) F(h(y - u_{k_2}))
%\quad \mbox{where} \quad u_k = (k-1/2)\del, \ k=1, \cdots, h.
\ee
where $u_k = (k-1/2)\del, \ k=1, \cdots, h$.
It is easy to see that $\Psi_{k_1, k_2} (x,y)= \Psi_{k_2, k_1} (y,x)$ 
and, for different pairs  $(k_1, k_2)$,  functions  $\Psi_{k_1, k_2} (x,y)$
have disjoint supports. Similar to the case of the piecewise constant graphon, 
consider an even  $L_0 \leq L/2$ and a set of vectors $\bom \in \{ 0,1\}^{L_0}$ with exactly $L_0/2$ nonzero entries.
%
By Lemma 4.10 of Massart (2007), there exists a subset $\calT$ of those vectors such that \fr{bombom} holds
for any $\bom, \bom' \in \calT$.
Let again $\calT_{k_1, k_2}$ be the copies of $\calT$ for $(k_1, k_2) \in \calK$ 
and denote $\tilde{\calT} = \prod_{(k_1,k_2) \in \calK} \calT_{k_1, k_2}$
where $\calK = \{(k_1, k_2):\ 1 \leq k_1 < k_2 \leq h\}$.
Then, 
\be \label{card11}
\log|\tilde{\calT}|= h(h+1)/2\, \log|{\calT}| \geq 0.04\, L_0 h^2. 
\ee
For any $\tilde{\bom} \in \tilde{\calT}$ and $l=L_0+1, \cdots, 2L_0$, define 
\begin{align} 
& \bv_1^{(\tilde{\bom})} (x,y) = \sqrt{L}/2; \label{bv_val} \\
& \bv_l^{(\tilde{\bom})} (x,y) =  
\al   \sum_{k_1 =1}^h \sum_{k_2 = k_1 + 1}^h \bom_{l-L_0}^{(k_1, k_2)}\, [\Psi_{k_1, k_2} (x,y) + \Psi_{k_2, k_1} (x,y)].
\nonumber 
%\quad l=L_0+1, \cdots, 2L_0.
 \end{align}   
It is easy to see that $\bv^{(\tilde{\bom})} (x,y) = \bv^{(\tilde{\bom})} (y,x)$ for any $x,y \in [0,1]$. 
Now we need to check that conditions \fr{A1} and \fr{A2} hold. 


Note that for any $x,y,x',y'$, due to \fr{kern_properties}, obtain
\begin{align}
& |\Psi_{k_1, k_2} (x,y)   - \Psi_{k_1, k_2} (x',y')|   \leq 
h^{-\nu_1} \lkv |F(h(x-u_{k_1})- F(h(x'-u_{k_1})| |F(h(y-u_{k_2})| \right. \nonumber \\
& \left. +  |F(h(y-u_{k_2})- F(h(y'-u_{k_2})| |F(h(x'-u_{k_1})| \rkv  
  \leq C_{\psi}  \lkv |x-x'| + |y-y'| \rkv^{\nu_1}, \label{ineqPsi}
\end{align}
where constant $C_{\psi}$ depends  only on $C_F$ and $\nu_1$.
Since functions $\Psi_{k_1, k_2} (x,y)$ have disjoint supports for different pairs of indices  $(k_1, k_2)$, 
the sums below have at most four nonzero terms. Then, \fr{ineqPsi} implies that 
\begin{align*}
& \left| \bv_l^{(\tilde{\bom})} (x,y) - \bv_l^{(\tilde{\bom})} (x',y')\right| 
\leq \al \sum_{k_1, k_2=1}^h   |\Psi_{k_1, k_2} (x,y)   - \Psi_{k_1, k_2} (x',y')|\\
& \leq 4 \al\,  C_{\psi}\  \lkv |x-x'| + |y-y'| \rkv^{\nu_1},
\end{align*}
so that \fr{A1} holds if $\al \leq K_1/(4   C_{\psi})$.  
%
Also, it is easy to check that, by \fr{kern_properties}, one has
$[\bv_l^{(\tilde{\bom})} (x,y)]^2 \leq C_v^2  \al^2 h^{-2\nu_1}$ where $C_v$ depends  only on $C_F$ and $\nu_1$.
Therefore, 
\bes
\sum_{l = L_0+1}^{2L_0} (l-1)^{2 \nu_2} [\bv_l^{(\tilde{\bom})} (x,y)]^2  \leq C_v^2 \al^2 h^{-2\nu_1} L_0^{2 \nu_2+1}.
\ees  
 Hence, both assumptions, \fr{A1} and \fr{A2} are valid provided
\be \label{rhoineq}
\al \leq \min \lkr K_1/(4   C_{\psi}),\ \sqrt{K_2}/C_v\, h^{\nu_1} L_0^{-(\nu_2+1/2)}\rkr.
\ee
Denote by $\bLam$ and $\bLam'$ the probability tensors corresponding to $\tilde{\bom}$ and 
$\tilde{\bom'}$ in $\tilde{\calT}$, respectively. 
Let $\tilde{\bv} (x,y)$ be a vector with  $\tilde{\bv}_1 (x,y) = 0$ and $\tilde{\bv}_l (x,y) = \bv_l (x,y)$
for $l\geq 2$. 
By \fr{H_assump} and \fr{bv_val}, since $\Psi_{k_1, k_2} (x,y)$ have disjoint supports,
we derive that for any $\tilde{\bom} \in \tilde{\calT}$ one has
\bes
\left| \Lam_{i,j,l}  - 1/2\right| \leq    \| \bH \tilde{\bv}  (x,y) \|_\infty
\leq  \|\tilde{\bv}  (x,y) \|_1/\sqrt{L} \leq C_v \al L_0 h^{-\nu_1}/\sqrt{L}.
\ees
Hence, $\bLam_{i,j,l}, \bLam'_{i,j,l} \in [1/4;3/4]$ provided
\be \label{rho_as1}
\al \leq \sqrt{L}\, h^{\nu_1}/(4 C_v  L_0).
\ee  
Then, by   Lemma~\ref{lem:Gao2015}, since each $\bom \in   \calT$ has exactly $L_0/2$ nonzero entries  
\begin{align*}
K (\PP_{\bLam}, \PP_{\bLam'}) & \leq    8 \|\bLam - \bLam'\|^2  
\leq 8 \al^2  L_0/2\ \sum_{i,j}^n \sum_{k_1, k_2}^h 4\, \lkv \Psi_{k_1, k_2} \lkr \frac{i}{n}, \frac{j}{n} \rkr \rkv^2\\
& =   16 \al^2  h^{-2\nu_1} L_0 \ \sum_{i,j}^n \sum_{k_1, k_2}^h F^2 (h(i/n-u_{k_1})) F^2 (h(j/n-u_{k_2})) \\
& =   16 \al^2  h^{-2\nu_1} L_0 \lkv  \sum_{i=1}^n \sum_{k=1}^h F^2(h(i/n-u_k))\rkv^2.
\end{align*}  
Here, 
\begin{align} \label{F_integral}
&  \sum_{i=1}^n \sum_{k=1}^h F^2(h(i/n-u_k))   = 
\sum_{k=1}^h \quad  \sum_{u_k-\del/2 <i/n \leq u_k+\del/2} F^2(h(i/n-u_k))   \\
& \approx   n \sum_{k=1}^h \int_{u_k-\del/2}^{u_k+\del/2} F^2(h(i/n-u_k)) dx 
   =   n \int_{-1/2}^{1/2}  F^2(z) dz,  \nonumber
\end{align} 
so that 
\be \label{KL11}
K (\PP_{\bLam}, \PP_{\bLam'}) \leq 16\, \|F\|_2^4\, \al^2  h^{-2\nu_1} n^2 L_0
\ee 
where $\|F\|_2$ is the $L^2$-norm of $F$.
On the other hand,  due to the first inequality in \fr{bombom} and \fr{F_integral}, obtain  
\begin{align*}
&\frac{\|\bLam - \bLam'\|^2}{n^2\, L} \geq    \frac{2 \al^2}{n^2\, L} \, \sum_{l=L_0+1}^{2L_0} \, \sum_{k_1 =1}^r \sum_{k_2 = k_1 +1}^r  
\sum_{i=1}^n \sum_{j=1}^n [\bom_l^{(k_1, k_2)} - {\bom'_l}^{(k_1, k_2)}]^2 [\Psi_{k_1, k_2} (i/n, j/n)]^2\\
&\geq  \frac{\al^2 h^{-2\nu_1} L_0}{4\, n^2 \, L} \lkv \sum_{i=1}^n \sum_{k=1}^h F^2(h(i/n-u_k)) \rkv^2 
\geq  \frac{\al^2 h^{-2\nu_1} L_0}{8\, L} \|F\|_2^4.
\end{align*} 
 Application of Theorem 2.5 of Tsybakov (2009) yields that
\fr{graph_lower_ineq} holds with 
$\Del(n,L) = C\, \al^2 h^{-2\nu_1} L_0/L$ provided $K (\PP_{\bLam}, \PP_{\bLam'}) \leq 1/8 \log|\tilde{\calT}|$,
which, due to \fr{card11} and \fr{KL11}, is guaranteed by 
\bes
4 \|F\|_2^4 \al^2  h^{-2\nu_1} n^2 L_0 \leq (1/8)\,  (0.04 L_0 h^2)/2
\ees
and leads to  the following restriction on $\al$:
\be \label{rho_as2}
\al \leq  h^{\nu_1 +1} n^{-1} /(40 \sqrt{2} \|F\|_2^2).
\ee
Set 
\be \label{hL0}
h = n^{\frac{1}{\nu_1+1}},\   L_0 = \min\lkr n^{\frac{2 \nu_1}{(\nu_1+1)(2 \nu_2 +1)}},\frac{L}{2} \rkr,\   
n_L = \lkr \frac{L}{2} \rkr^\frac{(\nu_1+1)(2 \nu_2 +1)}{2 \nu_1} 
\ee 
and consider two cases. 


If $n \leq n_L$, then $L_0$ is given by the first expression 
in \fr{hL0} and    inequalities \fr{rhoineq}, \fr{rho_as1} and \fr{rho_as2}  hold if $\al^2$ is a small enough 
absolute constant that depends on $\nu_1, \nu_2, K_1$ and $K_2$ only. In this case, 
$\Del(n,L) = C\, L^{-1}    n ^{-\frac{4 \nu_1 \nu_2}{(\nu_1 +1) (2\nu_2+1)}}$
and \fr{graph_lower_cases} is valid. 

If $n > n_L$, then $L_0=L/2$ and again all inequalities \fr{rhoineq}, \fr{rho_as1} and \fr{rho_as2}  hold if $\al^2$ is a small enough 
absolute constant that depends on $\nu_1, \nu_2, K_1$ and $K_2$ only.
In this case, $\Del(n,L) = C\,  n^{-\frac{2\nu_1}{\nu_1 +1}}$ which completes the proof.
 



%%%%%%%%%%%%%%%%%%%%%%%%%%%%%%%%%%%%%%%%%%%%%%%%%%%%%%%%%%%%%%%%%%%%%%%%%%%%%%%%%%%%%%%%%%%%%
%%%%%%%%%%%%%%%%%%%%%%%%%%%%%%%%%%%%%%%%%%%%%%%%%%%%%%%%%%%%%%%%%%%%%%%%%%%%%%%%%%%%%%%%%%%%% 
  

\subsection{Proofs of supplementary statements}
\label{sec:supplement}

In this section, we formulate and prove supplementary lemmas used in the proofs of other statements.
\\


%%%%%%%%%%%%%%%%%%%%%%%%%%%%%%%%%%%%%%%%%%%%%%%%%%%%%%%%%%%%%%%%%%%%%%%%%%%%%%%%%%%%%%%%%%%%%%%%%%%%%%%5


% lemma 2
\begin{lemma} \label{lem:bern_err}
Let $a_i$ be independent Bernoulli$(\te_i)$ variables and 
consider vectors $\ba$ and $\bte$ with components $a_i$ and $\te_i$,
respectively. Then, for any vector $\bz$ and any positive $t$ and   $\al$ one has
\be \label{expineq_bernoulli}
\PP \lkr 2 |\bz^T(\ba - \bte)| >  \af \| \bz \|^2  +  2t/\al \rkr \leq 2 e^{-t}, \quad
\EE \lkv \exp (\bz^T(\ba - \bte) \rkv \leq \exp(\|\bz\|^2/8).
\ee
\end{lemma}
 

\noindent
{\bf Proof. } Validity of the  Lemma follows from Hoeffding inequality (see, e.g., Massart (2007)).
\\

%%%%%%%%%%%%%%%%%%%%%%%%%%%%%%%%%%%%%%%%%%%%%%%%%%%%%%%%%%%%%%%%%%%%%%%%%%%%%%%%%%%%%%%%%%%%%%%%%%%%%%%%%



% lemma 3
\begin{lemma} \label{lem:subgaus_quadr_err}
Let $\bxi$ be a vector with independent components, $\EE \bxi = 0$ and such that for any vector 
$\bz$ and some $\sig>0$ one has
\be \label{subgaus_cond}
\EE \lkv \exp (\bz^T \bxi) \rkv \leq \exp(\|\bz\|^2 \sig^2/2).
\ee
Then, for any matrix  $\bA$ and any positive $t$   one has
\be \label{subgaus_expineq_quadr}
\PP \lfi \|\bA\bxi \|^2  > \sig^2 \lkr 2 \| \bA \|^2 +   3 \,\| \bA \|_{op}^2\, t \rkr \rfi \leq  e^{-t}.
\ee
\end{lemma}

\noindent
{\bf Proof. } Denote   $\bSig = \bA^T \bA$ and use Theorem 2.1 of Hsu \etal (2012).
Obtain, for any $t>0$,  
\bes
\PP \lkv \|\bA \bxi \|^2  > \sig^2 \lkr \Tr(\bSig) + 2   \sqrt{t\, \Tr(\bSig^2)} 
+2 t\, \|\bSig\|_{op}  \rkr \rkv \leq e^{-t}.  
\ees
Note that $\Tr(\bSig^2) \leq \|\bSig\|_{op} \Tr(\bSig)$ and $2 \sqrt{xy} \leq x+y$ for any positive $x$ and $y$.
In order to complete the proof, recall that $\|\bSig\|_{op} = \| \bA \|_{op}^2$ and $\Tr(\bSig) = \| \bA \|^2$.
\\


%%%%%%%%%%%%%%%%%%%%%%%%%%%%%%%%%%%%%%%%%%%%%%%%%%%%%%%%%%%%%%%%%%%%%%%%%%%%%%%%%%%%%%%%%%%%%%%%%%%%%%%%%

\begin{corollary} \label{corr:quadr_err}
Let $a_i$ be independent Bernoulli$(\te_i)$ variables and 
consider vectors $\ba$ and $\bte$ with components $a_i$ and $\te_i$,
respectively. Then, for any matrix  $\bA$ and any positive $t$   one has
\be \label{expineq_quadr}
\PP \lkr \|\bA (\ba - \bte)\|^2  > \frac{  \| \bA \|^2}{2} +  \frac{3t\,\| \bA \|_{op}^2}{4}\rkr \leq  e^{-t}.
\ee
\end{corollary}


\noindent
{\bf Proof. } Denote $\bxi = \ba-\bte$ and note that, due to \fr{expineq_bernoulli}, condition \fr{subgaus_cond}
holds with $\sig = 1/2$.
\\

%%%%%%%%%%%%%%%%%%%%%%%%%%%%%%%%%%%%%%%%%%%%%%%%%%%%%%%%%%%%%%%%%%%%%%%%%%%%%%%%%%%%%%%%%%%%%%%%%%%%%%%%%


% lemma 4
\begin{lemma} \label{lem:packing} {\bf (The packing lemma). }
Let $\calZ (m,n) \subseteq \calM (m,n)$ be a collection of clustering matrices.
Then, there exists a subset $S_{n,m} (r) \subset \calZ (m,n)$ such  that 
for $\bC_1, \bC_2 \in \calZ (m,n)$ one has 
$\| \bC_1 - \bC_2 \|_H =  \| \bC_1 - \bC_2 \|^2 \geq r$
and 
$\log |S_{n,m} (r)| \geq  \log|\calZ (m,n)| - r \log (n e m/r)$.
\end{lemma}

\noindent
{\bf Proof. } For any clustering matrix $\bC$ define the $r$-neighborhood of $\bC$ as 
\bes
\calB (\bC, r) = \lfi \tilbC \in \calZ (m,n):\ \|\tilbC - \bC\|_H \leq r \rfi.
\ees
Let $\tilS_{n,m}(r)$ be the covering set of $\calZ (m,n)$ and $|\tilS_{n,m}(r)|$
be the covering number, the smallest number of closed balls of radius $r$ whose union
covers $\calZ (m,n)$.  Let $|S_{n,m}(r)|$ be the packing number of $\calZ (m,n)$,
the largest number of points with the distance at least $r$ between them. 
It is known (see Pollard (1990), page 10) that
\be \label{pollard_ineq}
|\tilS_{n,m}(r)| \leq |S_{n,m}(r)| \leq |\tilS_{n,m}(r/2)|
\ee 
Note that $|\calZ (m,n)| \leq \sum_{\bC \in \tilS_{n,m}(r)} |\calB (\bC, r)| \leq 
|\tilS_{n,m}(r)| \max_{\bC \in \tilS_{n,m}(r)} |\calB (\bC, r)|$, so that
%
\be \label{cardF}
|\tilS_{n,m}(r)| \geq  |\calZ (m,n)| \Bigg/  \max_{\bC \in \tilS_{n,m}(r)} |\calB (\bC, r)| 
\ee 
and, also, 
\be \label{cardB}
|\calB (\bC, r)| \leq {n \choose r} m^r \leq \lkr \frac{ne}{r}\rkr^r m^r = \lkr \frac{n e m}{r}\rkr^r.
\ee
Now, combining \fr{pollard_ineq} -- \fr{cardB}, obtain
$\log |S_{n,m}(r)| \geq \log |\tilS_{n,m}(r)|  \geq \log |\calZ (m,n)| - r \log(n e m/r)$
which completes the proof.
\\



%%%%%%%%%%%%%%%%%%%%%%%%%%%%%%%%%%%%%%%%%%%%%%%%%%%%%%%%%%%%%%%%%%%%%%%%%%%%%%%%%%%%%%%%%%%%%%%%%%%%%%%%%


% lemma 5
\begin{lemma} \label{lem:cardinality}
Let $\ga m$ and $n/m$ be positive integers. Then, for $n_0 \geq 1$ and $\ga m \geq 2$, one has
\begin{align}
& \log \lfi (\ga n)! \rfi - \ga m \, \log \lfi \lkr n/m \rkr !\rfi \geq \frac{\ga n}{4} \log(\ga m); \label{AA1}\\
& \log \lfi {m \ga \choose n_0} \lkr \frac{n}{m} \rkr^{n_0} (n_0-1)!  \rfi \geq n_0 \log\lkr \frac{\ga n}{n_0} \rkr 
+ \frac{n_0}{2} \log \lkr \frac{n_0}{2}\rkr; \label{AA2}\\
& 6 \ga n \log(\ga m)  -  \ga n  \log(32\, m   \ga \, e) \geq 0. \label{AA3}
\end{align}
\end{lemma}


\noindent
{\bf Proof. } Note that due to \fr{inequalities}, one has 
\bes
A_1  = \log \lfi (\ga n)! \rfi - \ga m \, \log \lfi \lkr n/m \rkr !\rfi  
\approx \ga n \log (\ga m) - \frac{\ga m}{2} \log \lkr\frac{2 \pi n}{m}\rkr + \frac{1}{2} \log(2\pi \ga n).
\ees 
Consider a function
\bes
F(x) = \frac{3 \ga n}{4} \log x  - \frac{x}{2} \log \lkr \frac{2 \pi \ga n}{x} \rkr + \frac{1}{2} \log(2 \pi \ga n).
\ees 
It is easy to check that $F(1)=0$ and that 
$F'(x) =  0.75\, \ga n/x  - 0.5 \log(2 \pi \ga n/(x e))$. Replacing $2 \pi \ga n/(x e)$ in $F'(x)$ by $z$ and noting that the inequality
$3e/(4 \pi) z > \log z$ is true for any $z>0$, we confirm that $F'(x) >0$. so that $F(x) >0$ for any $x \geq 1$. 
Finally, in order to prove \fr{AA1}, observe that $A_1 = F(\ga m) +  \ga n \log(\ga m)/4$.


For the sake of proving \fr{AA2},  note that for every $n_0 \geq 1$ one has 
$\log[(n_0-1)!] \geq 0.5\, n_0 \log(n_0/2)$ and apply the first inequality in \fr{inequalities}.

The validity of  inequality \fr{AA3} follows from $\ga m \geq 2$ and the fact that $\log(64 e) < 6$.  
\\ 


 
%%%%%%%%%%%%%%%%%%%%%%%%%%%%%%%%%%%%%%%%%%%%%%%%%%%%%%%%%%%%%%%%%%%%%%%%%%%%%%%%%%%%%%%%%%%%%%%%%%%%%%%%%


% lemma 6
\begin{lemma} \label{lem:subgaussian}
Let $a_i$ be independent  Bernoulli$(\te_i)$ variables with $0\leq \te_i \leq \rho < 1/2$, $i=1, \cdots, n$. 
Denote $\xi_i = a_i - \te_i$  and 
\be \label{sum_subgaus}
\zeta = \frac{1}{\sqrt{n}} \ \sum_{i=1}^n \xi_i.
\ee
Then, $\zeta$ is a sub-Gaussian random variable with sub-Gaussian norm 
\be \label{subgaus}
\|\zeta\|_{\psi_2} = \sup_{p \geq 1} \lkr p^{-1/2}\, \EE |\zeta|^p\rkr^{1/p} \leq \frac{3 e^2}{2} \, \max \lkr \rho, \frac{1}{n} \rkr. %, \quad p \geq 1.
\ee
\end{lemma}
 


\noindent
{\bf Proof. } First, observe that we need to check inequality of the type \fr{subgaus} only for an even integer $p=2m$.
Recall that $\|\zeta\|_{\psi_2} \leq K_0$ provided 
$\EE |\zeta|^p \leq (K_0 \sqrt{p})^p = (K_0^2 p)^{p/2}$ for every $p \geq 1$. Suppose that  
 $\EE |\zeta|^{2m}  \leq (K_1 \sqrt{2m})^{2m} = (2 K_1^2 m)^m$ for every integer $m \geq 1$ and some  some $K_1 >0$.    
%
For any $p \geq 1$ choose $m$ such that $2m-2 < p \leq 2m$ which implies  $2m < p+2$. Then,
$$
\EE |\zeta|^p \leq \lkr \EE  \zeta^{2m} \rkr^{\frac{p}{2m}} \leq \lkr (2 K_1^2 m)^m \rkr^{\frac{p}{2m}} \leq \lkr   K_1^2 (p+2) \rkr^{\frac{p}{2}}
$$
Observing that $p+2 \leq 3p$ for $p\geq 1$, obtain that  $\EE |\zeta|^p \leq  [(\sqrt{3} K_1)^2 p]^{p/2}$. 
Hence, \fr{subgaus} is valid provided that for any $m \geq 1$
\be \label{2m_ineq}
\EE (\zeta)^{2m} = \EE \lkr \frac{1}{\sqrt{n}} \ \sum_{i=1}^n \xi_i \rkr^{2m} 
\leq  (e^2 m \rho_n)^m, \quad \rho_n= \max(\rho, 1/n). % ,\quad m \geq 1.
\ee
%
Note that 
\bes  
\EE (\zeta)^{2m} =  \sum \frac{\EE(\xi_{i_1}^{m_1} \ldots \xi_{i_l}^{m_l})}{n^m}
\ees
where the sum is taken over all positive integers $l, m_1, \ldots, m_l, i_1, \ldots, i_l$ such that $2m = m_1 + m_2 + m_l$
and $i_1, \ldots, i_l$ are distinct. Note that $\EE \xi_i =0$ for every $i$, hence, in the sum the terms with $m_l=1$ are 
equal to zero. Therefore, 
% 
\be \label{2m_ineq1}
\EE (\zeta)^{2m} \leq  \sum_{l}  S(m,l) P(n,l) \frac{\EE (\xi_{i_1}^{m_1} \ldots \xi_{i_l}^{m_l})}{n^m}
\ee
where $S(m,l)$ is the number of partitions of $2m$ unlabeled objects into $l$ distinct subparts of size at least two,
and $P(m,l)$ is the number of ways of choosing $l$ variables out of $n$ (order matters since powers can be different):
$$
S(m,l) = {2m-l-1 \choose l-1}, \quad P(n,l) = \frac{n!}{(n-l)!} \leq n^l.
$$
Here, $2m - l-1 \geq l-1$, so that $l \leq m$. Observe that due to
$$
\EE \xi_i^k =  \te_i(1-\te_i) [(-\te_i)^{k-1} + (1-\te_i)^{k-1}] \leq \rho,
$$ 
\fr{2m_ineq1} yields 
\bes 
\EE (\zeta)^{2m} \leq  \sum_{l=1}^{\min(m,n)}  {2m-l-1 \choose l-1} \frac{\rho^l}{n^{m-l}}.
\ees
By considering the cases $\rho \geq 1/n$ and $\rho < 1/n$ separately, it is easy to show that
$$
\frac{\rho^l}{n^{m-l}} \leq [\max(\rho, n^{-1})]^m  = \rho_{n}^m,
$$
so that 
\bes 
\EE (\zeta)^{2m} \leq \rho_{n}^m\ \sum_{l=0}^{\min(m,n)}  {2m-l  \choose l}  \leq 
\rho_{n}^m\ \sum_{l=0}^{\infty}  \frac{(2m)^l}{l!} \leq \rho_{n}^m\, e^{2m} = (e^2 \rho_{n})^m
\ees
which implies \fr{2m_ineq} and, hence, \fr{subgaus}.
\\

%%%%%%%%%%%%%%%%%%%%%%%%%%%%%%%%%%%%%%%%%%%%%%%%%%%%%%%%%%%%%%%%%%%%%%%%%%%%%%%%%%%%%%%%%%%%%%%%%%%%%%%5

\begin{corollary} \label{cor:subgaus}
Let $\ba \in [0,1]^n$ be a random vector with 
 independent Bernoulli$(\te_i)$ components. Let $\bxi = \ba - \bte$ and $\bZ \in \calM (M,N)$ be a clustering matrix
with $\bS^2 = \bZ^T \bZ = \diag(N_1, \cdots, N_M)$. Define $\boeta = \bS^{-1} \bZ^T \bxi$. 
Then, vector $\boeta$ has independent sub-Gaussian 
components with  
\bes
\| \boeta_k \|_{\psi_2}^2 \leq   \frac{3e^2}{2} \, \max( \|\te\|_{\infty}, N_k^{-1}).
\ees
\end{corollary}



\noindent
{\bf Proof. } Validity of the Lemma follows from the fact that each $\boeta_k$ is of the form \fr{sum_subgaus}
and is evaluated using a distinct set of components of vector $\bxi$.
\\

 
\vspace{4mm}

%%%%%%%%%%%%%%%%%%%%%%%%%%%%%%%%%%%%%%%%%%%%%%%%%%%%%%%%%%%%%%%%%%%%%%%%%%%%%%%%%%%%%%%%%%%%%%%%%%%%%%%%%


\begin{lemma} \label{lem:Gao2015} {\bf (Proposition 4.2 of Gao, Lu and Zhou (2015)) }
Let $\bte = \{ \te_{i,j}\} \in [0,1]^{n \times n}$. Let $\PP_{\te_{i,j}}$ denote the probability of
Bernoulli $(\te_{i,j})$ and the probability $\PP_{\bte}$ stands for the product measure 
$\PP_{\bte} = \otimes_{i,j} \PP_{\te_{i,j}}$.  Then, for any $\bte, \bte' \in [1/4, 3/4]^{n \times n}$ 
one has
$$
K(\PP_{\bte},\PP_{\bte'}) \leq 8 \, \sum_{i,j} (\te_{i,j} - \te'_{i,j})^2.
$$
\end{lemma}
 




 
 




 

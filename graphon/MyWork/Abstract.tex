%!TEX root = Main.tex

\begin{abstract}
 In neuroscience, it is not well understood how newborn neurons form a mature nervous system, due to the lack of observations. A recent study (Wan et al. 2019) made available a dataset of this functional maturation process on zebrafish. This novel data, however, introduce inherent challenges for the analysis of the formation of the nervous system.  First, the formation process is transient by nature.  The non-stationarity of the process makes the amount of data pale in comparison to the size of the neural network.  Moreover, combining observations on multiple subjects are not straightforward since the neural circuits are not identical across subjects.   In this talk, we propose a model for describing the emergence of a coordinated network from isolated nodes.  To this end, we adapt and generalize the stochastic block model for random graphs. The proposed method learns the transferable features across subjects, while allowing for individual variabilities. Briefly, the proposed method classifies nodes into different functional groups by identifying typical connecting behavior. We further employ the shape invariant models to handle the nodal delays due to neuron-specific delays.  We establish the consistency and minimax optimality of the proposed estimator.  We demonstrate the performance of our algorithm on simulation experiments, and on the real zebrafish dataset.
\end{abstract}
%!TEX root = Main.tex

\section{Simulation} \label{sec:simulation}


\begin{figure}[H]
\includegraphics[width=\textwidth]{../simulation/plots/conn_pattern}
\caption{ Estimated connecting patterns (ten trials). Dash line: average curves of estimated connecting patterns. Red line: true connecting patterns.}
\end{figure}



\begin{figure}[H]
\includegraphics[width=.8\textwidth]{../simulation/plots/violin_cpmr}
\caption{Clustering result.}
\end{figure}


\begin{figure}[H]
\begin{subfigure}{.49\textwidth}
\includegraphics[width=\textwidth]{../simulation/plots/overclus_a}
\caption{}
\end{subfigure}
\begin{subfigure}{.49\textwidth}
\includegraphics[width=\textwidth]{../simulation/plots/overclus_b}
\caption{}
\end{subfigure}
\begin{subfigure}{.49\textwidth}
\includegraphics[width=\textwidth]{../simulation/plots/overclus_d}
\caption{}
\end{subfigure}
\begin{subfigure}{.49\textwidth}
\includegraphics[width=\textwidth]{../simulation/plots/overclus_c}
\caption{}
\end{subfigure}
\caption{
2D representation by t-SNE.
(a) Aggregated point processes with  estimated three clusters.
(b) Aggregated point processes with estimated five clusters.
(c) Point processes incorporated clustering results from (a).
(d) Point processes incorporated clustering results from (b). }
\label{fig: explain overclustering}
\end{figure}










% \subsection*{Over-clustering does not always help}
% Possible reasons:
% \begin{itemize}
% 	\item Makes some clusters too small, the estimate for connecting patterns get worse.
% 	\item Spectral clustering does not keep hierarchical structure, over-clustering may get worse clustering results.
% 	\item Individual connecting patterns are similar after eliminating  time lags between groups, so different (but similar) groups might be merged together. 
% \end{itemize}






% \subsection*{Remark}
% \begin{enumerate}
% \item Using pdf can avoid getting large distance between curves with similar shapes but different peak heights
% \item Smoothing can cause non-identifiability of uniform distribution and normal distribution.
% \item A bigger size of clusters can lead to a better smooth result.
% \item When combining subjects, consider the biological clock?
% \item Spectral clustering does not keep hierarchical structure.
% \end{enumerate}

% In the first case we analyze the network with two types of nodes.
% Figure \ref{fig: nodes locations} shows the locations of 50 nodes, among which 4 are from type I and the rest 46 are from type II.
% The first type of nodes (type I) are generated uniformly from $[0.3, 0.7]\times[0.8, 5.2]$.
% The second type of nodes (type II) are generated uniformly in $[0,1]\times[0,6]$. 
% \\

% \noindent
% The network is developed during time period $[0,50]$.
% Two nodes are connected if the distance between them is less than $1$.
% The connecting time between two type II nodes is generated from uniform distribution $U(0,40)$.
% For the pair of nodes with one from type I and the other from type II, the connecting time is distributed as $N(5+\tau,1)$, where $\tau$ is the time delay caused by the type I node and is generated randomly from $U(0,30)$.





% The second case is with three clusters.
% The locations of nodes are displayed in Figure \ref{fig: nodes locations, case 2}.
% The connecting radius for type I node is set as $2$, and that for other nodes is set as $1$. 
% For a pair of type III nodes, the connecting time is generated from uniform distribution $U(0,30)$.
% For the pair of nodes with one from type II and the other from type III, 
% the connecting time is distributed as $N(5+\tau,1)$, where $\tau$ is the time delay caused by the type II node and is generated randomly from $U(0,5)$.
% For the pair of nodes with one from type I and the other from type II, the connecting time is distributed as $U(\tau,\tau+6)$, where $\tau$ is the time delay caused by the type II node and is generated randomly from $U(40,42)$.





%!TEX root = Main.tex

\section{Model} \label{sec:model}
% \subsection{Notations}


\subsection{Stochastic block model} % deleted the "undirected" setting

A set of $n$ nodes $\Gamma=\left\{ 1,\cdots,n \right\}$ is partitioned into $k$ clusters $\Gamma_1,\cdots,\Gamma_k$. 
The cluster of node $i$ is represented by $z_i\in \left\{ 1,\cdots,k \right\}$, and the vector of clusters is $\bz=\left( z_i \right)_{i=1}^n $. 
Define the adjacency matrix $\bA\in \left\{ 0,1\right\}^{n\times n}$ where  $A_{i,j}= 1$ if an edge is observed between node $i$ and node $j$ and $A_{i,j}=0$ otherwise.
We set ${A}_{i,i}\equiv 0$  for any $i=1,\cdots,n$, 
and assume that $A_{i,j}$'s are conditionally independent given the cluster vector $\bz$:
\begin{align*}
{A}_{i,j}|z_i=q,z_j=l \overset{ind}{\sim} \text{Bernoulli}({C}_{q,l}), \qquad i\neq j,
\end{align*}
where $\mathbf{C}\in [0,1]^{k\times k}$ denotes the  connecting probability matrix.




\subsection{Dynamic generalization of the stochastic block model}
% Things to be written in this section:
% Assume that the edges will not disappear once constructed,

% The integrated point process $N_{i,\cdot}(\cdot):=\sum_{j\neq i} N_{i,j}(\cdot)$ can be viewed as a realization of the intensity function $\lambda_{i,\cdot}(\cdot)=\sum_{j\neq i}\lambda_{i,j}(\cdot)$.
% For convenience, we denote $N_{i,\cdot}(\cdot)$ by $N_i(\cdot)$, and $\lambda_{i,\cdot}(\cdot)$ by $\lambda_{N_i}(\cdot)$. 


In a growing network where the edges appear over time, 
the edge between a pair of nodes $i$ and $j$ can be represented by $N_{i,j}(\cdot)$ with intensify function
\begin{align*}
\lambda_{i,j}(t)=\lambda_{z_i,z_j}(t), \qquad t\in[0,T],\quad i,j=1,\cdots,n, 
\end{align*}
where $[0,T]$ is overall time period,
$\lambda_{z_i,z_j}(\cdot)$ is the connecting intensity function between cluster $z_i$ and $z_j$. 
Similar to the stochastic block model, we set $\lambda_{i,i}(\cdot)\equiv 0$ for $i=1,\cdots,n$.
% Add more rigorous definition of point process and intensity function.


In practice, there are usually more restrictions to the network. 
Motivated by the neuronal network (see \cite{Wan2019} for detail) where neurons become active in different time and only connect with neighbor neurons, we incorporate the time delay and spatial location of each node and propose the following model:
\begin{align*}
\lambda_{i,j}(t)
=\lambda_{z_i,z_j}(t-\tau_{i,j})
\cdot \mathbf{1}_{\left\{ d_{i,j}\leq d^* \right\}}, 
\qquad t\in[0,T],\quad i,j=1,\cdots,n, 
\end{align*}
where $T$ and $\lambda_{z_i,z_j}$ is defined as before, $\tau_{i,j}$ is the time delay caused by both node $i$ and node $j$, $d_{i,j}$ is the spatial distance between node $i$ and $j$, and the node $i$ and $j$ can be connected only if $d_{i,j}\leq d^*$.
\\
For the convenience of estimation, we make the following assumptions.
\begin{assumption}\label{asp:time lag}
$\tau_{i,j}$ only depends on node $i$, that is,
$\tau_{i,j}=\tau_i $ for all $j\neq i, i=1,\cdots,n$.
\end{assumption}
% This also makes sense in practice because [XXX].


\noindent With Assumption \ref{asp:time lag}, 
we may consider the integrated point process ${N}_{i}(\cdot):=\sum_{j\neq i}N_{i,j}(\cdot)$
and write its intensity function as
\begin{align*}
\lambda_{N_i}(t) &= \sum_{l=1}^k \left( \lambda_{z_i,l}(t-\tau_i)\cdot \sum_{j\in\Gamma_l,j\neq i}\mathbf{1}_{\left\{ d_{i,j}\leq d^* \right\}} \right) \\
&=: \sum_{l=1}^k  \lambda_{z_i,l}(t-\tau_i)\cdot w_{i,l}.
\end{align*}
Here $w_{i,l}$ is the number of nodes from cluster $l$ that is in the neighborhood of node $i$.

\noindent We also assume that the nodes are distributed uniformly in the sense that $w_{i,l}$ and $w_{j,l}$ are identically distributed for any nodes $i, j$ from the same cluster.
More formally, we have the following assumption.

\begin{assumption}\label{asp:same distr}
For any $l=1,\cdots,k$, $\left\{ w_{i,l} \right\}_{i\in\Gamma_l}$ are i.i.d. with mean $\bar w_{z_i,l}$ and variance $\sigma^2<\infty$.
% $w_{i,l}=\bar w_{z_i,l}+\epsilon_{i,l}$ where $\left\{ \epsilon_{i,l} \right\}_{i\in\Gamma_l,l=1,\cdots,k}$ are i.i.d. random variables with mean $0$ and variance $\sigma^2<\infty$.
\end{assumption}
\noindent
By Assumption \ref{asp:same distr},
$\lambda_{N_i}(t+\tau_i)\overset{d}{=}\lambda_{N_j}(t+\tau_j)$ for node $i,j$ with $z_i=z_j$.
And hence we can define the mean intensify function for each group $\lambda_l(t)\overset{\triangle}{=}\mathbb{E}\lambda_{N_i}(t+\tau_i),i\in\Gamma_l$.
% for node $i,j$ in the cluster $l$ (i.e. $z_i=z_j=l$).
We aim at estimating the clusters $\mathbf{z}$, the mean intensity functions $\left\{ \lambda_{l}(\cdot) \right\}_{l=1}^k$, and the time delays $\left\{ \tau_i \right\}_{i=1}^n$.

% \subsubsection*{Minor comments}

% Hoeffding's inequality might be useful later. (If $\epsilon_{1},\cdots,\epsilon_d\overset{i.i.d}{\sim}\text{sub-G}(\tau_0)$ and $\mathbb{E}\epsilon_{i}=0$, then $\mathbb{P}(\langle a,\epsilon\rangle\geq t)\leq \exp \left\{ -t^2/(2\|a\|_2^2\tau_0^2) \right\}$ for any $a\in \mathbb{R}^d$.)

% Let $\mathbf{F}_{k\times k} = [f_{q,l}(\cdot)]_{q,l\in \left\{ 1,\cdots,k \right\}}$, $\mathbf W_{n\times k} = [w_{i,l}]_{i\in \left\{ 1,\cdots,n \right\}, l\in \left\{ 1,\cdots,k \right\}}$, $\mathbf{Z}\in \left\{ 0,1 \right\}^{n\times k}$ with $\mathbf Z_{i,l}=1$ if $z_i=l$ and $0$ otherwise. Then
% \begin{align*}
% \begin{bmatrix}
% \lambda_{N_1}(\cdot+\tau_1)\\
% \vdots\\
% \lambda_{N_n}(\cdot+\tau_n)
% \end{bmatrix}=\text{diag}\left( \mathbf{ZFW}^\top \right) .
% \end{align*}


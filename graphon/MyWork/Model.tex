%!TEX root = Main.tex

\section{Model} \label{sec:model}
% \subsection{Notations}


\subsection{Stochastic block model} 

A set of $n$ nodes $\Gamma=\left\{ 1,\cdots,n \right\}$ is partitioned into $k$ clusters $\Gamma_1,\cdots,\Gamma_k$. 
The cluster of node $i$ is represented by $z_i\in \left\{ 1,\cdots,k \right\}$, and the vector of clusters is $\mathbf{z}=\left( z_i \right)_{i=1}^n $. 
Define the adjacency matrix $\mathbf{A}\in \left\{ 0,1\right\}^{n\times n}$ where  $A_{i,j}= 1$ if an edge is observed between node $i$ and node $j$ and $A_{i,j}=0$ otherwise.
We set ${A}_{i,i}\equiv 0$  for any $i=1,\cdots,n$, 
and assume that $A_{i,j}$'s are conditionally independent given the cluster vector $\mathbf{z}$:
\begin{align*}
{A}_{i,j}|z_i=q,z_j=l \overset{ind}{\sim} \text{Bernoulli}({C}_{q,l}), \qquad i\neq j,
\end{align*}
where $\mathbf{C}\in [0,1]^{k\times k}$ denotes the  connecting probability matrix.




\subsection{Dynamic stochastic block model for point processes}


We consider a developing network where pairwise connections between $n$ nodes are observed during time interval $[0,T]$.
Each node belongs to one of $k$ groups denoted by $\Gamma_1, \cdots, \Gamma_k$, and the group membership for node $i$ is denoted by $z_i$.
% We assume that the connection between nodes are completely determined by their group memberships.
The connection between node $i$ and $j$ is modeled by a point process $N_{i,j}(\cdot)$ with conditional intensity depending only on their group memberships $z_i$ and $z_j$. 
Given $z_i=q, z_j=l$, the conditional intensify function of $N_{i,j}$ is
\begin{align*}
\lambda^*_{i,j}(t) = \lambda^*_{q,l}(t), 
\qquad t\in[0,T],\quad i,j=1,\cdots,n.
\end{align*}
Here $^*$ is used to denote that the density is conditional on the past.
Similar to the stochastic block model, we set $\lambda^*_{i,i}(\cdot)\equiv 0$ for $i=1,\cdots,n$.

Note that edges in the developing networks are considered as physical connections --- they do not disappear after constructed, so
one can observe at most one event (appearance of an edge) in each point process.
This allows us to define the event time $t_{i,j}$ in $N_{i,j}$, where $t_{i,j}=\infty$ if there is no event during $[0,T]$.
% For a realization $t$ of $N$ over $[0,T]$, the likelihood can be expressed as
% \begin{align*}
% L = \lambda^*(t) \exp \left( -\int_0^T \lambda^* (s) \text{d}s \right) 
% \end{align*}
The conditional distribution of $t_{i,j}$ given that there is an event during $[0,T]$ is then expressible in the form
\begin{align*}
p_{q,l}(t_{i,j}) \equiv p(t_{i,j}|N_{i,j}[0,T]=1) = \frac{L(t_{i,j};\lambda_{q,l}^*)}{\int_0^T L(u;\lambda_{q,l}^*) \text{d}u },
\end{align*}
where $L(t;\lambda^*)=\lambda^*(t) \exp \left( -\int_0^T \lambda^* (s) \text{d}s \right) $ is the likelihood of the realization $t_1=t$ of $N$ over $[0,T]$.
This simplifies our model to
\begin{align}
	t_{i,j}|t_{i,j}<\infty,z_i=q,z_j=l \sim p_{q,l}(\cdot).
\end{align}
Here we rule out the probability of having a connection between node $i$ and $j$, 
because it may vary drastically from node to node and therefore does not contribute to identifying clusters.

Let us now take into account the edge-specific time lags caused by the different active time of neurons.
Denoting the time lag for $t_{i,j}$ by $\tau_{i,j}$, our observation becomes
 $\tilde t_{i,j} = t_{i,j}+\tau_{i,j}$.
It is believed (?) in neuroscience that every connection is led by one of the two neurons, so the time lags are actually controlled by the active time of the leader neurons.
We incorporate the underlying structure of $\tau_{i,j}$ by making the following assumption
\begin{align}
\tau_{i,j} = v_i \mathbf{1}\{L_{q,l}=0\} + v_j \mathbf{1}\{L_{q,l}=1\}, \text{ given } i\in\Gamma_q, j\in\Gamma_l,
\label{eq: structure of tau}
\end{align}
where $\mathbf{v}_{n\times1}$ is a vector of relative active time satisfying $\min_{i\in\Gamma_q}v_i=0$, $L_{k\times k}$ is a matrix of relative leadership --- the edge time delay $\tau_{i,j}$ is governed by node $i$ if $L_{q,l}=0$, otherwise by node $j$.




\subsubsection*{Objective function}

With this model, we propose the following least-square loss function that measures the overall distance between each neuron and its group
\begin{align}
	\min_{\substack{\mathbf{z},\mathbf{f},\tau}} 
	\sum_{i\in[n]} 
	\left[\sum_{l\in[k]}
		w_{i,l}\cdot
		 \|f_{i, l} - p_{z_i,l}\|_2^2 \right],
	\label{eq: objective function 2}
\end{align}
where $ f_{i, l}(\cdot)$ is the kernel density estimation of sequence $\{\tilde t_{i,j}-\tau_{i,j}: \tilde t_{i,j}<\infty\}_{j\in\Gamma_l}$ with some proper bandwidth,
$w_{i,l}$ is the weight measuring the contribution of cluster $\Gamma_l$ to the distance between node $i$ and cluster $z_i$.







%!TEX root = Main.tex

\section{Model} %%% introcude $Gamma_a, a=1,\cdots,k$ somewhere in this section!
% \subsection{Notations}


\subsection{Stochastic block model} % deleted the "undirected" setting
A set of $n$ nodes $\Gamma=\left\{ v_1,\cdots,v_n \right\}$ is partitioned into $k$ clusters $\Gamma_1,\cdots,\Gamma_k$. For simplicity, we use the notation $i\in\Gamma_l$ to denote that $v_i\in\Gamma_l$. The cluster of node $v_i$ is represented by $z_i\in \left\{ 1,\cdots,k \right\}$, and the vector of clusters is $\bz=\left( z_i \right)_{i=1}^n $. 
Define the adjacency matrix $\bA\in \left\{ 0,1\right\}^{n\times n}$ where  $\bA_{i,j}= 1$ if an edge is observed between $v_i$ and $v_j$ and $\bA_{i,j}=0$ otherwise.
We set $\mathbf{A}_{i,i}\equiv 0$  for any $i=1,\cdots,n$, 
and assume that $\bA_{i,j}$'s are conditionally independent given the cluster vector $\bz$:
\begin{align*}
\mathbf{A}_{i,j}|z_i=q,z_j=l \overset{ind}{\sim} \text{Bernoulli}(\mathbf{C}_{q,l}), \qquad i\neq j,
\end{align*}
where $\mathbf{C}\in [0,1]^{k\times k}$ denote the  connecting probability matrix.



\subsection{Dynamic generalization of the stochastic block model}
Consider a growing dynamic network where edges appear over time. 
Assume that the edges will not disappear once constructed,
and that the observed point processes ${N}_{i,j}(\cdot)\in \left\{ 0,1 \right\}$ are independent realizations of intensity functions 
\begin{align*}
\lambda_{i,j}(t)=f_{z_i,z_j}(t-\tau_{i,j})\cdot g(d_{i,j}), \qquad t\in[0,T],\quad i,j=1,\cdots,n, 
\end{align*}
where $[0,T]$ is overall time period,
$\tau_{i,j}$ is the time lag of the edge from $v_i$ to $v_j$, 
 $d_{i,j}$ is the spatial distance between $v_i$ and $v_j$, $f_{z_i,z_j}(\cdot)$ is the connecting intensity function from cluster $z_i$ to $z_j$, and $g(\cdot)$ is a decreasing function that accounts for the decay of connection as the distance between any pair of nodes increases.
Similar to the stochastic block model, we set $\lambda_{i,i}(\cdot)\equiv 0$ for $i=1,\cdots,n$.
% Add more rigorous definition of point process and intensity function.
\\
The integrated point process $N_{i,\cdot}(\cdot):=\sum_{j\neq i} N_{i,j}(\cdot)$ can be viewed as a realization of the intensity function $\lambda_{i,\cdot}(\cdot)=\sum_{j\neq i}\lambda_{i,j}(\cdot)$.
For convenience, we denote $N_{i,\cdot}(\cdot)$ by $N_i(\cdot)$, and $\lambda_{i,\cdot}(\cdot)$ by $\lambda_{N_i}(\cdot)$. 

In many applications, the edge from $v_i$ to $v_j$ is determined only by the activity of $v_i$. 
So we may have the following assumption.
\begin{assumption}\label{asp:time lag}
$\tau_{i,j}$ only depends on $v_i$, that is,
$\tau_{i,j}=\tau_i $ for all $j\neq i, i=1,\cdots,n$.
\end{assumption}
\noindent With Assumption \ref{asp:time lag}, $\lambda_{N_i}(\cdot)$ can be written as
\begin{align*}
\lambda_{N_i}(t) &= \sum_{l=1}^k \left( f_{z_i,l}(t-\tau_i)\cdot \sum_{j\in\Gamma_l,j\neq i}g(d_{i,j}) \right) \\
&=: \sum_{l=1}^k  f_{z_i,l}(t-\tau_i)\cdot w_{i,l}.
\end{align*}
Here $w_{i,l}$ measures the overall distance between node $v_i$ and cluster $\Gamma_l$. 
For example, if the nodes represent neurons and are clustered by cell types, $w_{i,l}$ represents the distance between $v_i$ and its neighbor cells  which belong to cell types $l$.
We assume that the cell types are distributed uniformly in the sense that $w_{i,l}$ and $w_{j,l}$ are identically distributed for any $i, j$ such that $z_i=z_j$.
More formally, we have the following assumption.
 % which means $w_{i,l}=\bar w_{z_i,l}+\epsilon_{i,l}$ where $\left\{ \epsilon_{i,l} \right\}_{i\in\Gamma_l}$ are i.i.d. random variables with mean zero.

\begin{assumption}\label{asp:same distr}
% We assume $\left\{ w_{i,l} \right\}_{i\in\Gamma_l, l=1,\cdots,k}$ are i.i.d. random variables with mean $\bar w_{Z_i,l}$ and variance $\sigma^2<\infty$.
$w_{i,l}=\bar w_{z_i,l}+\epsilon_{i,l}$ where $\left\{ \epsilon_{i,l} \right\}_{i\in\Gamma_l,l=1,\cdots,k}$ are i.i.d. random variables with mean $0$ and variance $\sigma^2<\infty$.
\end{assumption}
\noindent
By Assumption \ref{asp:time lag} and \ref{asp:same distr}, $\lambda_{N_i}(t)=\lambda_{z_i}(t-\tau_i)+\sum_{l=1}^kf_{z_i,l}(t-\tau_i)\epsilon_{i,l}$ for $t\in[0,T]$, where $\lambda_{z_i}(t):=\sum_{k=1}^lf_{z_i,l}(t)\cdot\bar w_{z_i,l}$.

\subsubsection*{Minor comments}
% Hoeffding's inequality might be useful later. (If $\epsilon_{1},\cdots,\epsilon_d\overset{i.i.d}{\sim}\text{sub-G}(\tau_0)$ and $\mathbb{E}\epsilon_{i}=0$, then $\mathbb{P}(\langle a,\epsilon\rangle\geq t)\leq \exp \left\{ -t^2/(2\|a\|_2^2\tau_0^2) \right\}$ for any $a\in \mathbb{R}^d$.)

Let $\mathbf{F}_{k\times k} = [f_{q,l}(\cdot)]_{q,l\in \left\{ 1,\cdots,k \right\}}$, $\mathbf W_{n\times k} = [w_{i,l}]_{i\in \left\{ 1,\cdots,n \right\}, l\in \left\{ 1,\cdots,k \right\}}$, $\mathbf{Z}\in \left\{ 0,1 \right\}^{n\times k}$ with $\mathbf Z_{i,l}=1$ if $z_i=l$ and $0$ otherwise. Then
\begin{align*}
\begin{bmatrix}
\lambda_{N_1}(\cdot+\tau_1)\\
\vdots\\
\lambda_{N_n}(\cdot+\tau_n)
\end{bmatrix}=\text{diag}\left( \mathbf{ZFW}^\top \right) .
\end{align*}

